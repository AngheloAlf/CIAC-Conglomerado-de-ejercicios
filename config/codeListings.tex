\usepackage{listings}
\usepackage{newfloat}
\usepackage{adjustbox}

\renewcommand{\lstlistingname}{Código}

% Archivo de texto plano
\lstdefinestyle{archivo}{
  backgroundcolor=\color{white},
  frame=none,
  %basicstyle=\small\normalfont, 
  %basicstyle=\normalfont,
  basicstyle=\small\ttfamily,
  %basicstyle=\small\fontfamily{cmr}\selectfont,
  %basicstyle=\fontfamily{fvm}\selectfont,
  %basicstyle=\small\mdseries,
  %basicstyle=\small\textsf,
  %basicstyle=\small\fontfamily{put}\selectfont,
  %basicstyle=\small\ttfamily,
  %basicstyle=\fourier,
  %basicstyle=\footnotesize,
  %moredelim=[is][\bfseries]{[*}{*]},
  xrightmargin=5pt,
  %keepspaces=true,
  tabsize=2,
  columns=fixed
}

% Consola
\lstdefinestyle{consola}{
  backgroundcolor=\color{bggray},
  basicstyle=\small\ttfamily,
  frame=single,
  moredelim=[is][\bfseries]{[*}{*]},
  xrightmargin=5pt,
  tabsize=2
}

% Python
\lstdefinestyle{mypy}{
  language=python,
  backgroundcolor=\color{bggray},
  basicstyle=\ttfamily\small\color{orange!70!black},
  frame=L,
  keywordstyle=\bfseries\color{green!40!black},
  commentstyle=\itshape\color{purple!40!black},
  identifierstyle=\color{blue},
  stringstyle=\color{red},
  numbers=left,
  showstringspaces=false,
  xrightmargin=5pt,
  xleftmargin=10pt,
  tabsize=2
}


% Python simple
\lstdefinestyle{pythonSimple}{
  language=python,
  backgroundcolor=\color{bggray},
  basicstyle=\ttfamily\small\color{black!70!black},
  %frame=L,
  keywordstyle=\bfseries\color{green!40!black},
  commentstyle=\itshape\color{purple!40!black},
  identifierstyle=\color{blue},
  stringstyle=\color{red},
  showstringspaces=false,
  %xrightmargin=0pt,
  %xleftmargin=0pt,
  tabsize=44,
  otherkeywords={>>> },
  breaklines=true,
  columns=fullflexible,
  linewidth=\columnwidth
}

%%%% Esto lo agregó Miguel para los códigos en VBA de INF130
\lstdefinestyle{VBAStyle}{
  language={[Visual]Basic},
  backgroundcolor=\color{bggray},
  basicstyle=\ttfamily\small\color{black!70!black},
  %frame=L,
  keywordstyle=\bfseries\color{green!40!black},
  commentstyle=\itshape\color{purple!40!black},
  identifierstyle=\color{blue},
  stringstyle=\color{red},
  showstringspaces=false,
  xrightmargin=0pt,
  xleftmargin=0pt,
  tabsize=2,
  otherkeywords={>>> }
}
%%%%%% Fin de agregado 23/09/2020


% \inputArchivo{rutaArchivo}[nombreParaMostrar]{ruta/al/archivo.txt}
% Agrega un archivo de texto plano desde la ruta especificada.
% El argumento `nombreParaMostrar` es opcional.
\newcommand{\inputArchivo}[2][]%
{
\begin{center}
    \begin{tabular}{|l|}
        \multicolumn{1}{c}{\texttt{#1}}\\
	    \hline
	    \lstinputlisting[style=archivo]{#2}\\
		\hline
    \end{tabular}
\end{center}
}


% \inputArchivoNumerado{rutaArchivo}[nombreParaMostrar]{ruta/al/archivo.txt}
% Agrega un archivo de texto plano desde la ruta especificada, y agrega el número de cada linea a un costado.
% El argumento `nombreParaMostrar` es opcional.
\newcommand{\inputArchivoNumerado}[2][]%
{
\begin{center}
    \begin{tabular}{|l|}
        \multicolumn{1}{c}{\texttt{#1}}\\
	    \hline
	    \lstinputlisting[style=archivo,numbers=left]{#2}\\
		\hline
    \end{tabular}
\end{center}
}

% \inputConsola{ruta/al/archivo}
% Muestra un archivo de texto con el formato consola
\newcommand{\inputConsola}[1]{\lstinputlisting[style=consola]{#1}}


% \inputPython{ruta/al/archivo}{nombreParaMostrar}
% Muestra un archivo con formato Python.
% Agrega numeración a las líneas y tiene como encabezado "Código X: nombreParaMostrar".
\newcommand{\inputPython}[2]%
{
\lstinputlisting[style=mypy, caption=\texttt{#2}] {#1}
}


% \inputPythonSimple{ruta/al/archivo}
% Muestra un archivo con formato Python
\newcommand{\inputPythonSimple}[1]%
{
\lstinputlisting[style=pythonSimple] {#1}
}

%%%% Se agrega función para agregar VBA como comando

\newcommand{\inputVBA}[2]
{
\lstinputlisting[style=VBAStyle, caption=\texttt{#2}] {#1}
}
