\subsection*{Diccionarios}

\begin{itemize}
    \item Un diccionario es un tipo de dato desordenado el cual permite asociar pares de valores, uno siendo la ``llave'' y el otro su ``valor''
    
    \item Para crear un diccionario vació puede usar \{ \} o la función \textit{dict()}. Para crear un diccionario con elementos por defecto, debe hacerlo con \{ \} y separando cada llave y su valor asociado con \textit{:}
    \inputPythonSimple{Resumen_diccionarios/Consola/01_01.txt}

    \item Para obtener un valor que esta en la llave \textit{l} de un diccionario \textit{d}, debe usar $d[l]$
    \inputPythonSimple{Resumen_diccionarios/Consola/01_02.txt}

    \item Para agregar elementos, simplemente se asigna el valor a la llave correspondiente:
    \inputPythonSimple{Resumen_diccionarios/Consola/01_03.txt}

    Cabe destacar que si se asigna un valor a una llave que ya tenia un elemento, el elemento viejo se elimina y se guarda el nuevo (No se pueden tener llaves repetidas).

    \item Para borrar una llave:
    \inputPythonSimple{Resumen_diccionarios/Consola/01_04.txt}

    \item Se puede usar el ciclo \textit{for}, donde la variable \textit{i} sera las llaves del diccionario:
    \inputPythonSimple{Resumen_diccionarios/Consola/01_05.txt}

    \item Para iterar según los valores y no las llaves, se usa \textit{diccionario.values()}
    \inputPythonSimple{Resumen_diccionarios/Consola/01_06.txt}

    \item Para iterar según las llaves y los valores a la vez, se usa \textit{diccionario.items()}:
    \inputPythonSimple{Resumen_diccionarios/Consola/01_07.txt}

    \item El operador \textit{n in d} indica si la variable \textit{n} es una llave del diccionario \textit{d}
    \inputPythonSimple{Resumen_diccionarios/Consola/01_08.txt}
\end{itemize}
