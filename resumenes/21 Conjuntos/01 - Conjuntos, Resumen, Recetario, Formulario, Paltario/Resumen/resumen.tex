\section*{Resumen, Recetario, Formulario, Paltario, etc}

\subsection*{Conjuntos}

\begin{itemize}
\item Un conjunto es un tipo de dato que permite almacenar valores no repetidos y de forma desordenada

\item Para inicializar un conjunto use:
\begin{lstlisting}[style=consola]
conjunto={datos} #o bien
conjunto=set(datos)
\end{lstlisting}

\item Para agregar un elemento al conjunto use la función \textit{conjunto.add(elemento)}:
\begin{lstlisting}[style=consola]
>>> conjunto = {8, 9, "abc"}
>>> conjunto.add("def")
>>> print conjunto
set([8, 9, "abc", "def"])
\end{lstlisting}
Tener en consideración que set([]) es solamente una notación para indicar que es un conjunto, los elementos son los que están dentro de los corchetes []

\item No tienen elementos repetidos:
\begin{lstlisting}[style=consola]
>>> conjunto = {1, 9, 1, 5, 5, 1, 9}
>>> print conjunto
set([9, 5, 1])
\end{lstlisting}

\item Solo se pueden agregar elementos inmutables a conjuntos, es decir, enteros (\textit{int}), flotantes (\textit{float}), strings (\textit{str}), tuplas (\textit{tuple}), booleanos (\textit{bool}), etc.

\item A diferencia de listas y tuplas, los conjuntos no tienen orden, de modo que no se pueden obtener elementos a través de un índice.

\item Para usar los elementos de un conjunto, generalmente se usa el ciclo for:
\begin{lstlisting}[style=consola]
>>> conjunto = {1, 9, 1, 5, 5, 1, 9}
>>> for i in conjunto:
        print i
9
5
1
\end{lstlisting}

%%\item La función \textit{len} entrega la cantidad de elementos en el conjunto
%%    \begin{lstlisting}[style=consola]
%%    >>> conjunto = {1, 9, 1, 5, 5, 1, 9}
%%    >>> print len(conjunto)
%%    3
%%    \end{lstlisting}

\item El operador \textit{in} permite saber si un elemento en especifico se encuentra en el conjunto:
\begin{lstlisting}[style=consola]
>>> conjunto = {1, 9, 1, 5, 5, 1, 9}
>>> 5 in conjunto
True
>>> 3.14 in conjunto
False
>>> "abc" not in conjunto
True
\end{lstlisting}

\item La función \textit{conjunto.remove(elemento)} elimina el elemento del conjunto. Si el elemento no esta en el conjunto, tirara un error:
\begin{lstlisting}[style=consola]
>>> conjunto = {1, 9, 1, 5, 5, 1, 9}
>>> conjunto.remove(1)
>>> print conjunto
set([9, 5])
\end{lstlisting}

\item Como en matemáticas, podemos obtener la intersección y unión de conjuntos, esto se hace con los operadores \& y | respectivamente:
\begin{lstlisting}[style=consola]
>>> a = {1, 9, 5}
>>> b = {1, 422, 8, 3.14}
>>> print a & b
set([1])
>>> print a | b
set([1, 9, 5, 422, 8, 3.14])
\end{lstlisting}

\item El operador \textit{a - b} (diferencia de conjuntos) entrega un conjunto que contiene elementos de a quitando los que ya se encuentran en b:
\begin{lstlisting}[style=consola]
>>> a = {1, 9, 5}
>>> b = {1, 422, 8, 3.14}
>>> print a - b
set([9, 5])
\end{lstlisting}

\item El operador \textit{a \textasciicircum b} entrega la diferencia simétrica, es decir, los elementos que están en a o en b, pero no en ambos:
\begin{lstlisting}[style=consola]
>>> a = {1, 9, 5}
>>> b = {1, 9, 3.14}
>>> print a ^ b
set([3.14, 5])
\end{lstlisting}

\item El operador \textit{a < b} es lo mismo que preguntar si a es subconjunto de b:
\begin{lstlisting}[style=consola]
>>> a = {1, 9}
>>> b = {1, 9, 3.14}
>>> print a < b
True
\end{lstlisting}

\item El operador \textit{a <= b} es equivalente a preguntar si es subconjunto o si son iguales:
\begin{lstlisting}[style=consola]
>>> a = {1, 9}
>>> b = {1, 9, 3.14}
>>> print {1, 9, 3.14} < b
False
>>> print {1, 9, 3.14} <= b
True
\end{lstlisting}

\end{itemize}

\newpage
