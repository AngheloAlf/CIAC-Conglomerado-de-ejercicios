\section{Manejo de datos numéricos}

Existen dos tipos de datos que representan números: los enteros de tipo \texttt{int} como \texttt{-1},\texttt{0},\texttt{1313} y los flotantes (o reales) del tipo \texttt{float} como \texttt{3.1415}, \texttt{2.718}, \texttt{2.0}.

Una operación entre enteros dará como resultado un número entero, una operación mixta dará como resultado un número flotante y una operación entre flotantes dará como resultado un número flotante. Por ejemplo

\begin{lstlisting}[style=consola]
>>>3/4
0
>>>3+1.0
4.0
>>>3.0/4
0.75
\end{lstlisting}

Los datos que se ingresan a través del comando \texttt{raw\_input()} son transformados a string por Python, por lo que es conveniente transformarlos a entero o flotante dependiendo de nuestra intención:

\begin{lstlisting}[style=consola]
entero=int(raw_input('Ingrese numero entero: '))
flotante=float(raw_input('Ingrese numero flotante'))
\end{lstlisting}

El escribir \texttt{variable=str(raw\_input( ... ))} es repetir un proceso, provoca risas en los ayudantes y/o profesores correctores y hace que Guido Van Rossum (creador del lenguaje) llore por las noches.


Considerando esto trabaje en:
\begin{itemize}
    \item Un programa que pida 3 números e informe del promedio de ellos
    \begin{lstlisting}[style=consola]
    Ingrese primer numero: 5
    Ingrese segundo numero:9.7
    Ingrese tercer numero:-4
    El promedio de los numeros es 3.567
    \end{lstlisting}
    
    \item Un programa que, recibiendo una hora específica, indique cuanto tiempo falta para otra hora ingresada en formato HH:MM (omita el hecho de que HH o MM quede con sólo un número, problemas de este tipo los abordaremos conociendo estructuras condicionales)
    
    \begin{lstlisting}[style=consola]
    Ingrese las horas actuales:17
    Ingrese los minutos actuales:45
    Ingrese las horas siguientes:23
    Ingrese los minutos siguientes:00
    Son las 17:45 y para que sean las 23:0 queda 5:15 
    \end{lstlisting}
\end{itemize}
