\section{Manejo de strings}

Los datos que representan texto son llamados \texttt{strings} y se representan mediante comillas simples como \texttt{'Programar es genial'}, o dobles como \texttt{"\ El CIAC es el mejor lugar para estudiar"}.
\\
Los strings se pueden unir o concatenar usando el símbolo \texttt{+} de la siguiente forma

\begin{lstlisting}[style=consola]
>>>'Practicare Python todas las semanas ' + 'para aprobar progra'
'Practicare Python todas las semanas para aprobar progra'
\end{lstlisting}

o bien usando variables, nótese el uso de espacios en la concatenación

\begin{lstlisting}[style=consola]
>>>tutor_1='Pancho'
>>>tutor_2='Miguel'
>>>print tutor_1+' y '+tutor_2+' '+'son los mejores'
Pancho y Miguel son los mejores
\end{lstlisting}

Para conocer la cantidad de letras que tiene cierto string se usa el comando \texttt{len}

\begin{lstlisting}[style=consola]
>>>palabra='Intensivo'
>>>len(palabra)
9
\end{lstlisting}

Para ingresar datos se usa el comando \texttt{raw\_input("Texto indicativo")} el cual recibe una entrada que puede ser guardada en una variable

\begin{lstlisting}[style=consola]
variable=raw_input('Ingrese dato: ')
\end{lstlisting}

Teniendo esto en cuenta desarrolle:
\begin{itemize}
    \item Un programa que pida un nombre y un apellido y los imprima por pantalla
    \begin{lstlisting}[style=consola]
    Ingrese nombre: Miguel
    Ingrese apellido: Godoy
    Su nombre es Miguel y su apellido Godoy
    
    \end{lstlisting}
    
    \item Un programa que pida una palabra e informe cuantas letras tiene
    
    \begin{lstlisting}[style=consola]
    Ingrese una palabra: Aprender
    La palabra Aprender tiene 8 letras.
    \end{lstlisting}
    
\end{itemize}