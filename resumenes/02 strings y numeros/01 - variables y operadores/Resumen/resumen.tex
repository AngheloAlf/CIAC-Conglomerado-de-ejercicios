\section{"Pequeño" torpedo}


\begin{itemize}
    \item Variables \\
En python (por ahora) tenemos 4 tipos de datos o variables: enteros (int), flotantes o reales (float), palabras (strings) y valores de verdad (boolean).
\begin{lstlisting}[style=consola]
>>> a = 1
>>> a
1
>>> str(a)
'1'
>>> int(a)
1
>>> float(a)
1.0
>>> bool(a)
True
\end{lstlisting}

    \item Operadores \\
Cada uno de estos tipos de datos tiene distintas operaciones:
    \begin{itemize}
        \item Palabras o String\\
            Aca tenemos suma de palabras y repeticion de $n$ veces (multiplicacion):
            \begin{lstlisting}[style=consola]
>>> "Hola" + "Pedro" # Suma
'HolaPedro'
>>> "Ole" * 4 # Multiplicacion
'OleOleOleOle'
>>> "Ole" * 4.0
Traceback (most recent call last):
  File "<stdin>", line 1, in <module>
TypeError: can't multiply sequence by non-int of type 'float'
            \end{lstlisting}
        \item Numeros (Enteros o Flotantes)\\
            Aqui contamos con todos los operadores matematicos comunes:
            \begin{lstlisting}[style=consola]
>>> 1+2 #Suma
3
>>> 1/2 #Division
0
>>> 1/2.0
0.5
>>> 1*2 #Multiplicacion
2
>>> 1*2.0
2.0
>>> 1-2 #Resta
-1
>>> 1%2 #Modulo o resto
1
>>> 2**3.0 #Potencia
8.0
>>> 2**(1.0/2)
1.4142135623730951
            \end{lstlisting}


        \item Valores de verdad\\
            Para los valores de verdad tenemos comparadores logico/matematicos, los que siempre nos entregaran un valor de verdad (o valor logico/booleano):
            \begin{lstlisting}[style=consola]
>>> 1 < 2 # Menor
True
>>> 1 > 2.0 # Mayor
False
>>> "Hola" < 2
False
>>> 1 <= 1 # Menor igual
True
>>> True or False # O
True
>>> True and False # Y
False
>>> 1 == '1' # Igualdad
False
>>> 1 != 1.0 # Desigualdad
False
>>> (1 < 2) and ( (2+3) == 5 ) # Combinacion de los anteriores
True
            \end{lstlisting}
    \end{itemize}
    \item Funciones\\
        \begin{itemize}
            \item print\\
                Sirve para mostrar datos por la pantalla(consola/IDLE)
            \begin{lstlisting}[style=consola]
>>> print 1 < 2
True
>>> print "Hola" + "Gato"
HolaGato
>>> print 1 + "Chao"
Traceback (most recent call last):
  File "<stdin>", line 1, in <module>
TypeError: unsupported operand type(s) for +: 'int' and 'str'
>>> print "Marta, sos la numero ", 1
Marta, sos la numero  1
            \end{lstlisting}           
        \end{itemize}

\end{itemize}
