\documentclass{article}
\usepackage[utf8]{inputenc}

\title{CIAC Conglomerado de ejercicios}
\author{\textit{CIAC Pythonic Team}}
\date{}

\begin{document}

\maketitle

Hay 2 grandes categorías actualmente en el conglomerado, \texttt{Ejercicios} y \texttt{Resúmenes}

Tanto \texttt{Ejercicios} como \texttt{Resúmenes} están categorizados en 12 categorías:

\begin{itemize}
    \item Diagramas de flujo
    \item Strings y numeros
    \item Condiciones y ciclos
    \item Funciones
    \item Patrones comunes
    \item Listas y tuplas
    \item Ciclo For
    \item Diccionarios
    \item Funciones avanzadas
    \item Conjuntos
    \item Procesamiento de texto
    \item Manejo de archivos
\end{itemize}

Dentro de cada categoría debe haber una carpeta por cada ejercicio. En esta carpeta debe haber una guía y ojala una pauta con el ejercicio en cuestión. La guía debe poder compilar en Overleaf.

Las categorías están ordenadas de tal forma que cuando llegas a un ejercicio de una categoría, se da por hecho que el estudiante ya entiende las categorías anteriores y se puede usar materia de aquellas categorías.

Los resúmenes deben cumplir lo anterior, es decir, deben poder compilar y ser impresos inmediatamente.

En caso de creer que este orden y categorización es mala, sientase libre de reordenar todo de modo que sea mas entendible, ojala documentando lo realizado (aunque no es mandatorio).

\texttt{\# TODO}: Actualizar los ejercicios a Python 3.

\end{document}
