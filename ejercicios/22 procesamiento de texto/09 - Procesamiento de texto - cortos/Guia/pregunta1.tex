\section{Procesamiento de texto}
A continuación se muestran distintas tablas, con una columna de entrada y otra de salida. Su tarea es crear una función o un programa que transforme el string de la entrada en el de salida.

\subsection{Cambiar y borrar elementos}
\begin{table}[h]
	\centering
	\label{replace}
	\begin{tabular}{|l|l|}
		\hline
		\textbf{Entrada}              & \textbf{Salida}                \\ \hline
		'la tele se apaga'            & 'lo tolo so opogo'             \\ \hline
		'Miguel es el mejor del CIAC' & 'Miguel es el mejor del mundo' \\ \hline
		'Ajispirijou'                 & 'Ajosporijou'                  \\ \hline
		'no tengo barra espaciadora'  & 'notengobarraespaciadora'      \\ \hline
	\end{tabular}
\end{table}

\subsection{Separar strings en listas}

\begin{table}[h]
	\centering
	\label{split}
	\begin{tabular}{|l|l|}
		\hline
		\textbf{Entrada}                 & \textbf{Salida}                                       \\ \hline
		'hola como estas?'               & set({[}'hola','como','estas'{]})                      \\ \hline
		'14,15,16,17\textbackslash n'                  & {[}14,15,16,17{]}                                     \\ \hline
		'a\#3\#b\#5\#c\#7'               & \{'a':3, 'b':5, 'c':7\}                               \\ \hline
		'oh mira4no gracias4un dos tres' & ( ('ohmira', 6), ('nogracias', 9), ('undostres', 9) ) \\ \hline
		'Miguel Godoy\#22\#4,3,12\textbackslash n'     & {[}'Miguel',22,{[}4, 3, 12{]} {]}                     \\ \hline
	\end{tabular}
\end{table}

\subsection{Unir elementos}
\begin{table}[h]
	\centering
	\label{format y join}
	\begin{tabular}{|l|l|}
		\hline
		\textbf{Entrada}                    & \textbf{Salida}                                                 \\ \hline
		{[}'panqueques','manjar','pizza'{]} & 'Hoy desayune panqueques con manjar y en la noche cenare pizza' \\ \hline
		{[}12, 4, 7{]}                      & 'notas\#12,4,7\#semestre1'                                     \\ \hline
	\end{tabular}
\end{table}