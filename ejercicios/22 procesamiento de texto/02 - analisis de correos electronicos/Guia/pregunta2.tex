\section{Análisis de Correos Electrónicos}

  La empresa RawInput S.A. 
  desea hacer una segmentación de sus clientes
  según su ubicación geográfica.
  Para esto, 
  analizará su base de datos de correos electrónicos 
  con el fin obtener información
  sobre el lugar de procedencia de cada cliente.

  En una dirección de correo electrónico,
  el dominio es la parte que va después de la arroba,
  y el TLD es lo que va después del último punto.
  
  Por ejemplo, 
  en la dirección 
  \texttt{fulano.de.tal@alumnos.usm.cl},
  el dominio es 
  \texttt{alumnos.usm.cl}
  y el TLD es
  \texttt{cl}.

  Algunos TLD no están asociados a un país,
  sino que representan otro tipo de entidades.
  Estos TLD genéricos son los siguentes:
  
  \begin{lstlisting}[style=consola]
  genericos = {'com', 'gov', 'edu', 'org', 'net', 'mil'}
  \end{lstlisting}
  
  \begin{enumerate}
  \item
    Escriba la función 
    \texttt{obtener\_dominios(correos)}
    que reciba como parámetro 
    una lista de correos electrónicos,
    y retorne la lista de todos los dominios,
    sin repetir, 
    y en orden alfabético.
  \item
    Escriba la función 
    \texttt{contar\_tld(correos)}
    que reciba como parámetro la lista de correos electrónicos,
    y retorne un diccionario 
    que asocie a cada TLD 
    la cantidad de veces que aparece en la lista. 
    No debe incluir los TLD genéricos.
  \end{enumerate}
  
  \begin{lstlisting}[style=consola]
  >>> [*c = ['fulano@usm.cl', 'erika@lala.de', 'li@zi.cn', 'a@a.net', *]
  ... [*     'gudrun@lala.de', 'otto.von.d@lorem.ipsum.de', 'org@cn.de.cl', *]
  ... [*     'yayita@abc.cl', 'jozin@baz.cz', 'jp@foo.cl', 'dawei@hao.cn', *]
  ... [*     'pepe@gmail.com', 'ana@usm.cl', 'polo@hotmail.com', 'fer@x.com', *]
  ... [*     'ada@alumnos.usm.cl', 'dj@foo.cl', 'jan@baz.cz', 'd@abc.cl'] *]

  >>> [*obtener_dominios(c)*]
  ['abc.cl', 'alumnos.usm.cl', 'baz.cz', 'cn.de.cl', 'foo.cl',
   'hao.cn', 'lala.de', 'lorem.ipsum.de', 'usm.cl', 'zi.cn']

  >>> [*contar_tld(c)*]
  {'cz': 2, 'de': 3, 'cn': 2, 'cl': 8}
  \end{lstlisting}