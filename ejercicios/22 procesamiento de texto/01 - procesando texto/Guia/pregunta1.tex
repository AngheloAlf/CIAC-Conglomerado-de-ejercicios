\section{Funciones para Procesar Texto}

  Una de las técnicas más antiguas para
  proteger información sensible era el uso de
  tablas de transposición,
  en las cuales se cambiaban las letras de los mensajes
  por otras,
  según una tabla de conversión,
  dejando el mensaje original ilegible.
  Podemos representar una tabla de este tipo por medio
  de un diccionario como el que sigue:
  
  \begin{lstlisting}[style=consola]
  >>> Tablav = {'a': 'i', 'e': 'u', 'i': 'a', 'o':'e', 'u':'o'}
  \end{lstlisting}
  
  Si intercambiamos las letras como indica el diccionario,
  la frase: ``esta es la historia de un caballo''
  se transforma en:
  ``usti us li hasterai du on cibille''.
  
  \begin{enumerate}
  \item
    Programe la función
    \texttt{cifrar\_texto(texto, tabla)}
    que reciba un texto en formato
    \texttt{string} y
    retorne el texto con las vocales intercambiadas
    según el diccionario \texttt{tabla}.
  \item
    Otra técnica usada para asegurar los textos es
    cambiar palabras por otras.
    Considere el diccionatio:
    
    \begin{lstlisting}[style=consola]
    >>> Tabla = {'moneda': 'calabaza',
                 'pablo' : 'estrecho',
                 'botin' : 'servilleta',
                 'robo'  : 'abrazo'}
    \end{lstlisting}
    
    Programe la función
    \texttt{cifrar\_palabras(texto, tabla)}
    que reciba un \texttt{string} de texto
    y realice los reemplazos definidos en el diccionario tabla.
    
    \begin{lstlisting}[style=consola]
    >>> [*t = 'el robo en el palacio de la moneda \n  *]
        [*    fue un exito. el botin lo tiene pablo.'*]
    >>> [*cifrar_palabras(t, Tabla)*]
    el abrazo en el palacio de la calabaza fue un exito.
    el servilleta lo tiene estrecho.
    \end{lstlisting}
  \item
    Considere una lista de receptores del mensaje,
    como la que sigue:
    
    \begin{lstlisting}[style=consola]
    receptores = [('Pedro', 'Hernandez'),
                  ('Pietro', 'Morales'),
                  ('Sandro', 'Maureira')]
    \end{lstlisting}
    
    Programe la función
    \texttt{envios\_codificados(texto, receptores, tablavocal, tabla)}
    que aplique los cifrados anteriores al mensaje \texttt{texto}
    y retorne una lista de textos con el nombre del receptor
    al comienzo.
    Utilice la función \texttt{format} para crear la lista de textos.
    
    \begin{lstlisting}[style=consola]
    >>> [*t = 'el robo en el palacio de la moneda \ *]
        [*    fue un exito. el botin lo tiene pablo.'*]
    >>> [*envios_codificados(t, receptores, Tablav, Tabla)*]
    ['Estimado Pedro, Hernandez. ul ibrize un ul pilicae
     du li cilibizi fou on uxate. ul survalluti le taunu ustruche.',
     'Estimado Pietro, Morales. ul ibrize un ul pilicae
     du li cilibizi fou on uxate. ul survalluti le taunu ustruche.', 
     'Estimado Sandro, Maureira. ul ibrize un ul pilicae 
     du li cilibizi fou on uxate. ul survalluti le taunu ustruche.']
    \end{lstlisting}
  \end{enumerate}
  
  \pagebreak[4]