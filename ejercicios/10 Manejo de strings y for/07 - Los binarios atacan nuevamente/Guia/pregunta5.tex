%Pregunta 5

\section{Los binarios atacan nuevamente}

Como ya sabes como convertir un numero binario a un numero comprensible para todos, te preguntaras como hacer el proceso inverso, representar un numero como una secuencia binaria.

Para hacer esto, debes dividir un numero entre 2, y tomar su resto y anotarlo, luego el numero que ya tienes dividido, lo vuelves a dividir y anotas su resto, asi sucesivamente hasta que el numero quede como 0, y finalmente, la secuencia que tenias guardada debes invertirla y tendras tu secuencia binaria.

Mira el siguiente ejemplo que convierte el numero 13 en binario:
Nota: En la columna de la izquierda se anota el resultado de la division del numero, y en la columna de la derecha queda el resto de la division del numero.
\begin{center}
    \begin{tabular}{|c|c|}
        \hline
        \textbf{13} & \textbf{2} \\
        \hline
         & \\
        \hline
        6 & 1 \\
        \hline
        3 & 0 \\
        \hline
        1 & 1 \\
        \hline
        0 & 1 \\
        \hline
    \end{tabular}
\end{center}
\begin{center}
    La secuencia (leida de arriba a abajo) quedaria como $1011$, pero hay que recordar invertirla, asi que quedaria $1101$
\end{center}

Como veras, se divide el numero 13 entre 2, la parte entera de la division, 6, lo anotamos en la columna de la izquierda, y el resto de la division, 1, lo anotamos en la columna de la derecha, y asi sucesivamente hasta que la parte entera de la division nos da 0, ordenamos los restos obtenidos, invertimos el orden y nos da nuestro resultado.

Sabiendo esto, cree la funcion \texttt{numero\_a\_binario(numero)} que reciba un numero y retorne un string que contenga la secuencia binaria que lo representa.

\begin{lstlisting}[style=consola]
>>> numero_a_binario(25)
11001
\end{lstlisting}

\begin{lstlisting}[style=consola]
>>> binario_a_numero(6)
110
\end{lstlisting}
