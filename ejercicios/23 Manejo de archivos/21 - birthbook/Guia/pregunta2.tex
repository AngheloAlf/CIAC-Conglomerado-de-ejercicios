\section{Birthbook}

La red social birthbook requiere desarrollar una aplicación para que un usuario pueda enviar un saludo de cumpleaños de manera automática a todos sus amigos que se encuentren de cumpleaños. La información se encuentra en el archivo \texttt{amigos.txt} con la siguiente estructura: \texttt{nombre apellido; sexo;edad; grado de amistad;fecha\_cumpleanios}. El sexo puede ser M o F. El grado de amistad va de 1 a 5 (1 es poco amigos y 5 es muy amigo). La fecha del cumpleaños tiene siempre el formato que se muestra en el siguiente ejemplo.

\begin{table}[h]
\centering
\caption{amigos.txt}
\begin{tabular}{|l|}
\hline
\texttt{Juan Perez;M;5;23 de noviembre del anio 1990    } \\
\texttt{Marcela Soto;F;1;21 de octubre del anio 1992    } \\
\texttt{Andrea Olivares;F;3;13de enero del anio 2000     }\\
\texttt{Pedro Gonzalez;M;2;23 de noviembre del anio 2010} \\ \hline
\end{tabular}
\end{table}

Asuma que el archivo tiene muchos registros, no sólo los del ejemplo. Además los usuarios del archivo tienen registrado sólo un nombre y un apellido. Por otro lado, existe un archivo \texttt{saludos.txt} que tiene 5 saludos dependiendo del grado de amistad. La estructura es: \texttt{grado de amistad;mensaje}

\begin{table}[h]
\centering
\caption{saludos.txt}
\begin{tabular}{|l|}
\hline
\texttt{1;Feliz Cumpleanios {0}, suerte!} \\
\texttt{2;Feliz Cumpleanios {0} que lo pases bien!} \\
\texttt{3;Feliz Cumpleanios {0} que lo pases excelente en tu dia.}\\
\texttt{4;Amigo {0}, felicidades en tus {1} anios, espero sea un dia especial.} \\
\texttt{Amigazo {0}, celebraremos tus {1} anios como corresponde.}\\ \hline
\end{tabular}
\end{table}

Donde {0} es el nombre del amigo y {1} es el número de años (presente sólo en mensajes 4 y 5)

\begin{itemize}
    \item[a)] Escriba la función \texttt{leer\_amigos(archivo\_amigos)} que retorne una lista de tuplas de cada uno de los datos del archivo \texttt{archivo\_amigos} con la siguiente estructura:

\begin{lstlisting}[style=consola]
>>> leer_amigos('amigos.txt')
[('Juan Perez/M/5', '23-nov-1990'), ('Marcela Soto/F/1', '21-oct-1992'), 
('Andrea Olivares/F/3','13-ene-2000'),('Pedro Gonzalez/M/2','23-nov-2010')]
\end{lstlisting}

    \item[b)] Escriba la función \texttt{cumpleaneros(archivo\_amigos, archivo\_saludos, fecha)} que genere un archivo llamado \texttt{saludos-FECHA.txt}, donde FECHA corresponde a la fecha con el formato dia\_mes (por ejemplo si fecha es '10 de enero de 2012', el arhcio debería llamarse saludos -10\_ene.txt). El archivo debe contener el saludo, dependiendo del nivel de amistad, de los amigos que están de cumpleaños en la fecha \texttt{fecha}. Note el formato que tiene el parámetro \texttt{fecha} en el ejemplo:
    
    \begin{lstlisting}[style=consola]
>>>cumpleaneros('amigos.txt','saludos.txt',23 de noviembre 2012')
    \end{lstlisting}
    Esto debería generar un archivo como el siguiente:
    
\begin{table}[h]
\centering
\caption{saludos-23\_nov.txt}
\begin{tabular}{|l|}
\hline
\texttt{Amigazo Juan Perez, tus 22 anios lo celebraremos como corresponde} \\
\texttt{Feliz Cumpleanios Pedro Gonzalez que se cumplan todos tus deseos} \\ \hline
\end{tabular}
\end{table}
La función no debe retornar nada, sólo generar el archivo.
\end{itemize}