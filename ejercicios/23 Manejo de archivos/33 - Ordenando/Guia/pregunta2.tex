\section{Ordenando}

El siguiente codigo esta desordenado y sin indentar, su misión sera ordenarlo de modo que el codigo haga lo que su descripción dice.

\begin{itemize}

\item Ordene la funcion \texttt{promedio\_ramo(archivo, ramo)}, la cual recibe el nombre de un archivo y el nombre de un ramo. La estructura del archivo es \texttt{nombre\_alumno;ramo;nota1,nota2,...}. La funcion retorna el promedio de notas de \texttt{ramo}.

\begin{lstlisting}[style=consola]
for linea in arch:
sumaNotas += float(nota)
return sumaNotas/total
notas = linea[-1].split(";")
arch = open(archivo)
def promedio_ramo(archivo, ramo):
total = 0.0
arch.close()
total += 1.0
sumaNotas = 0.0
for nota in notas:
linea = linea.strip().split("#")
if linea[1] == ramo:
\end{lstlisting}

\end{itemize}
