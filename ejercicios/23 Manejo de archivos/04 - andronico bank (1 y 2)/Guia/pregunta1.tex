\section{Andrónico Bank (C3 CC 2015-II)}

  Andrónico Bank es un banco muy humilde
  que hasta hace poco usaban solo papel y lápiz
  para manejar toda la información de sus clientes,
  también humildes.
  Como una manera de mejorar sus procesos,
  Andrónico Bank quiere utilizar un sistema computacional
  basado en Python.
  Por eso se traspasa la información de sus clientes
  a un archivo de texto,
  indicando su rut,
  nombre y clase cliente.
  el archivo \texttt{clientes.txt}
  es un \emph{ejemplo} de lo anterior.
  A usted se le pide lo siguiente:
  
  \begin{lstlisting}[style = consola, caption = \texttt{clientes.txt}]
  9234539-9;Sebastian Davalos;VIP
  11231709-k;Choclo Delano;Pendiente
  5555555-6;Sebastian Pinera;VIP
  9999999-k;Gladis Maryn;RIP
  12312312-1;Michel Bachelet;VIP
  8888888-8;Companero Yuri;Estandar
  7987655-1;Sergio Estandarte;RIP
  \end{lstlisting}
  
  \begin{itemize}
  \item[a)]
    Escriba la función
    \texttt{buscar\_clientes(archivo, clase)}
    que reciba como parámetros el nombre del archivo
    y una clase,
    y retorne un diccionario con los rut
    de los clientes como llaves
    y los nombres como valor de todos los clientes
    pertenecientes a la clase entregada como parámetro.
    
    \begin{lstlisting}[style = consola]
    >>> [*buscar_clientes('clientes.txt', 'Pendiente') *]
    {'11231709-k': 'Choclo Delano'}
    \end{lstlisting}
  \item[b)]
    Escriba una función
    \texttt{dar\_credito(archivo, rut)}
    que reciba como parámetros
    el nombre del archivo de clientes
    y el \texttt{rut} de un cliente,
    y que retorne \texttt{True}
    si éste es \emph{VIP} o 
    \texttt{False} si no lo es.
    Si no encuentra el cliente
    la función retorna \texttt{False}.
    
    \begin{lstlisting}[style = consola]
    >>> [*dar_credito('clientes.txt', '9999999-k') *]
    False
    \end{lstlisting}
  \item[c)]
    Escriba una función
    \texttt{contar\_clientes(archivo)}
    que reciba como parámetros el nombre del archivo
    de clientes y que retorne un diccionario
    con la cantidad de clientes de cada clase
    en el archivo.
    
    \begin{lstlisting}[style = consola]
    >>> [*contar_clientes('clientes.txt')*]
    {'VIP: 3, 'Pendiente': 1, 'RIP': 2, 'Estandar':1}
    \end{lstlisting}
  \end{itemize}