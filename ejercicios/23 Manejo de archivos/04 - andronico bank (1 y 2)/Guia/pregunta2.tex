\section{Andrónico Bank 2 (C3 CC 2015-II)}

  Complementando la pregunta anterior,
  se le solicita:
  
  \begin{itemize}
  \item[a)]
    Escriba la función
    \texttt{nuevo\_cliente(archivo, rut, nombre, clase)}
    que reciba como parámetro el nombre del archivo de clientes
    y el rut,
    nombre y clase de un nuevo cliente.
    la función debe agregar el nuevo cliente al final del archivo.
    Esta función retorna \texttt{None}.
    
    \begin{lstlisting}[style = consola]
    >>> [*nuevo_cliente('clientes.txt', '2121211-2', 'Sergio Lagos', 'VIP')*]
    >>> 
    \end{lstlisting}
  \item[b)]
    Escriba la función
    \texttt{actualizar\_clase(archivo, rut, clase)}
    que reciba como parámetro el nombre del
    \texttt{archivo} de clientes,
    el \texttt{rut} de un cliente y una nueva \texttt{clase}.
    La función debe modificar la clase del cliente
    con el \texttt{rut} indicado,
    cambiándola por \texttt{clase} en el \texttt{archivo}.
    Esta función retorna \texttt{True} si logra
    hacer el cambio o \texttt{False} si no encuentra
    al cliente con el \texttt{rut} indicado.
    
    \begin{lstlisting}[style = consola]
    >>> [*actualizar\_clase('clientes.txt', '9234539-9', 'Estandar')*]
    True
    \end{lstlisting}
  \item[c)]
    Escriba una función
    \texttt{filtrar\_clientes(archivo, clase)}
    que reciba como parámetros el nombre del archivo de clientes
    y una clase de cliente.
    La función debe crear un archivo
    \texttt{clientes\_[clase].txt}
    con los rut y los nombres de los clientes
    pertenecientes a esa clase.
    Note que el archivo debe ser nombrado según la
    \texttt{clase} solicitada.
    Esta función retorna \texttt{None}.
    
    \begin{lstlisting}[style = consola]
    >>> [*filtrar_clientes('clientes.txt', 'VIP')
    >>>
    \end{lstlisting}
    
    \begin{lstlisting}[style = consola, caption = clientes\_VIP.txt]
    5555555-6;Sebastian Pinera
    12312312-1;Michel Bachelet
    2121211;Sergio Lagos
    \end{lstlisting}
  \end{itemize}