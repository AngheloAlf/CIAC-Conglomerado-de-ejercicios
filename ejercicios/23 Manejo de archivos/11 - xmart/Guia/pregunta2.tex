\section{Xmart}

  La empresa de videojuegos Xmart
  maneja los juegos en venta y preventa
  en dos archivos diferentes con la siguiente estructura:
  \texttt{titulo;empresa},
  como por ejemplo los siguientes:
  
  \begin{lstlisting}[style=consola, caption=\texttt{venta.txt}]
  starcraft ii;blizzard
  the last of us;naughty dog
  \end{lstlisting}
  
  \begin{lstlisting}[style=consola, caption=\texttt{preventa.txt}]
  uncharted 4;naughty dog
  the witcher 3;cd projekt RED
  sims 4;EA
  \end{lstlisting}
  
  Los archivos anteriores son ejemplos,
  pueden contener más juegos y no necesariamente
  se llaman \\
  \texttt{venta.txt} y
  \texttt{preventa.txt}.
  
  \begin{itemize}
  \item[a)]
    Desarrolle la función
    \texttt{agregar\_venta(nombre\_archivo, datos)},
    que reciba como parámetro una variable de tipo string
    con el nombre del archivo de ventas y los datos del juego
    como una lista.
    Debe agregar al final del archivo el juego ingresado,
    retornando \texttt{True}.
    En caso de que el nombre del juego ya se encuentre
    en el archivo,
    no lo guarda y simplemente retorna
    \texttt{False}.
    
    \begin{lstlisting}[style=consola]
    >>> [*agregar_venta('venta.txt', [    *]
      [*'assassins creed iV', 'ubisoft']) *]
    True
    \end{lstlisting}
  \item[b)]
    Desarrolle la función
    \texttt{preventa\_a\_venta(archivo\_preventa, titulo)}
    que reciba como parámetro tres variables de tipo string,
    una con el nombre del archivo con las ventas,
    otra con las preventas y la última con el título de un juego.
    La función crea un nuevo archivo de preventas
    (a cuyo nombre original se le antepone la palabra
    \texttt{nueva}),
    eliminando el juego y agregándolo en el archivo venta,
    retornando \texttt{True}.
    En caso de no existir en preventa,
    no lo guarda y simplemente retorna \texttt{False}.
    
    \begin{lstlisting}[style=consola]
    >>> [*preventa_a_venta('venta.txt', 'preventa.txt', 'sims 4')*]
    True
    \end{lstlisting}
  \item[c)]
    Desarrolle la función
    \texttt{buscar\_juegos(archivo\_venta, archivo\_preventa,} \\
    \texttt{empresa)}
    que recibe el nombre del archivo de ventas,
    de preventas y el nombre de una empresa de juegos.
    La función retorna una lista de tuplas,
    donde la tupla posee la siguiente estructura
    (nombre del juego, archivo donde se encuentra).
    Guíese por el ejemplo.
    
    \begin{lstlisting}[style=consola]
    >>> [*buscar_juegos('ventas.txt', 'preventa.txt', 'naughty dog')*]
    [('the last of us', 'venta'), ('uncharted 4', 'preventa')]
    \end{lstlisting}
  \end{itemize}