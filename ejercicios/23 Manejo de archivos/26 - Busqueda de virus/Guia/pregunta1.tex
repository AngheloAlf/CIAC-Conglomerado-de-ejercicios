\section{Búsqueda de Virus}

Una prestigiosa empresa de seguridad computacional, programa algunas funciones para encontrar palabras dentro de archivos de texto. Lamentablemente, creen que programar borracho es entretenido y eficaz, por lo que todas sus líneas de código, en un acto de estupidez, fueron mezcladas.

Ayude a estos personajes a ordenar su código para que funcione.

\begin{itemize}
    \item[a.] La función \texttt{esta\_palabra(archivo,palabra)} que reciba un string \texttt{palabra} y retorne True o False dependiendo si la palabra está o no en \texttt{archivo}.
    
    \begin{lstlisting}[style=consola]
if palabra in linea:
def esta_palabra(archivo,palabra):
for linea in arch:
arch=open(archivo)
return False
arch.close()
return True
arch.close()
    \end{lstlisting}
    
    \item[b.] La función \texttt{encontrar\_virus(archivos,malignos)} que reciba una lista de nombres de archivo (sin la extensión .txt) y una lista de strings \texttt{malignos} que contiene palabras consideradas como virus. Esta función retorna un conjunto con los nombres de los archivos que contienen virus.
\begin{lstlisting}[style=consola]
for linea in arch:
arch=open(archivo+'.txt')
arch.close()
def encontrar_virus(archivos,malignos):
return infectados
for palabra in malignos:
infectados.add(archivo+'.txt')
if palabra in linea:
infectados=set()
for archivo in archivos:
\end{lstlisting}
\end{itemize}

