\documentclass[spanish, fleqn]{scrartcl}
\usepackage[utf8]{inputenc}
\usepackage{babel}
\usepackage[headheight=47pt, paper=a4paper, top=3cm, left=2cm, right=2cm]{geometry}
\usepackage{tikz}
\usepackage{CIACcustom}
\usepackage{fourier}
\usepackage{amsmath, amsthm}
\usepackage{listings}
\usepackage{multicol}
\usepackage{fancyhdr}
\usepackage[urlcolor=blue, colorlinks]{hyperref}
\usepackage{booktabs,tabularx}
\usepackage{float}

\newcolumntype{L}[1]{>{\hsize=#1\hsize\raggedright\arraybackslash}X}%
\newcolumntype{R}[1]{>{\hsize=#1\hsize\raggedleft\arraybackslash}X}%
\newcolumntype{C}[2]{>{\hsize=#1\hsize\columncolor{#2}\centering\arraybackslash}X}%

\pagestyle{fancy}
\fancyhf{}
\rhead{\pgfimage[width=2.5cm]{imagenes/logo-ciac.png}}
\chead{
  Apoyos Intensivos Pauta N° 23 (modificada)\\
  IWI-131 Semestre I-2017 \\
  CIAC Casa Central
}
\lhead{\pgfimage[width=2.5cm]{imagenes/logo-usm.jpg}}
\rfoot{\LaTeXe / CIAC 2017 / M.G.}
\lfoot{\thepage}

\renewcommand{\ttdefault}{pcr}

%%% listings settings:
\definecolor{bggray}{rgb}{0.95,0.95,0.95}
\lstdefinestyle{consola}{
  backgroundcolor=\color{bggray},
  basicstyle=\small\ttfamily,
  frame=single,
  moredelim=[is][\bfseries]{[*}{*]},
  xrightmargin=5pt
}

\lstdefinestyle{mypy}{
  language=python,
  backgroundcolor=\color{bggray},
  basicstyle=\ttfamily\small\color{orange!70!black},
  frame=L,
  keywordstyle=\bfseries\color{green!40!black},
  commentstyle=\itshape\color{purple!40!black},
  identifierstyle=\color{blue},
  stringstyle=\color{red},
  numbers=left,
  showstringspaces=false,
  xrightmargin=5pt,
  xleftmargin=10pt
}

\newtheorem{CIACdef}{Definición}

\begin{document}
\vspace*{-0.4cm}

\section{Sistema Solar}

  A continuación se indica el código con las funciones
  solicitadas.
  
    \lstinputlisting[
    style  = mypy,
    caption= \texttt{sistema solar.py}]{Code/planetas.py}
    
\section{Certamen anterior}

    \lstinputlisting[
    style = mypy,
    caption= \texttt{pauta3.py}]{Code/pauta3.py}
    
\paragraph{Comentarios}
\begin{itemize}
    \item Las primeras dos funciones pueden ser acopladas dentro de las otras
    \item Si bien el problema pide un programa, el separar los mini-problemas, o requisitos del algoritmo en funciones hace mas entendible la lectura
\end{itemize}
    
\section{Banco de Animales}

    \lstinputlisting[
    style  = mypy,
    caption= \texttt{animales.py}]{Code/p3.py}
    


\end{document}