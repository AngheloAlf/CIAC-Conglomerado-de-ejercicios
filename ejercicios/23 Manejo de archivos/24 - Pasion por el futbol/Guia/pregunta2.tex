\section{Pasión por el fútbol}

Para el campeonato de fútbol pronto a jugar en la universidad, CIAC inscribirá a su equipo Ni por CIACaso. El único problema es que existen muchos jugadores y equipos dentro de la unidad, por lo que se jugó un campeonato interno para elegir al equipo más bueno.
 \\ \\
Los datos de los partidos jugados en la copa CIAC se guardan en un archivo de texto donde cada línea tiene los resultados de la forma \texttt{equipo\_1;equipo\_2;marcador\_1;marcador\_2}. 
\\ \\
Esto es sumamente difícil de leer por lo que se le pide a usted ayudar en manejar estos datos.
\\ \\
Se quiere hacer un diccionario cuyas llaves sean los nombres de equipos y cuyos valores sean tuplas de la forma (PJ,PG,PP,PE,Pts) donde PJ son partidos jugados, PG partidos ganados, PP partidos perdidos, PE partidos empatados y Pts los puntos obtenidos. Entiéndase que por cada partido ganado se suman 3 puntos, por empatado 1 punto y ninguno por partido perdido.
\\ \\
El archivo es el siguiente:

\begin{lstlisting}[style=consola]
Archivo: resultados.txt
===================
progra;matfis;1;2
matfis;quimica;3;2
progra;matfis;5;2
matfis;quimica;1;5
matfis;quimica;0;3
matfis;progra;0;3
quimica;progra;2;3
progra;quimica;5;3
quimica;matfis;4;1
\end{lstlisting}

Ayudaría mucho que el programa imprima los datos de la siguiente manera

\begin{lstlisting}[style=consola]
EQUIPO: (PJ,PG,PP,PE,PTS)
========================
quimica: (6, 3, 3, 0, 9)
matfis: (7, 2, 5, 0, 6)
progra: (5, 4, 1, 0, 12)
\end{lstlisting}