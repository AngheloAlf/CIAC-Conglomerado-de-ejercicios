\section{Selección de ayudante para CIAC}
CIAC cada semestre abre sus puertas para poder postular al cargo de tutor académico del ramo de Programización. Este cargo requiere mucha dedicación y manejo de la materia, por lo que los alumnos con muy baja prioridad o muy alta quedan principalmente excluidos. Idealmente se busca a un ayudante con la prioridad más cercana a 7000 y promedio de notas más cercano a 100.\\

Los datos de los alumnos postulantes se guardaron en un archivo de texto.\\
\begin{lstlisting}[style=consola]
[*alumnos.txt*]
=============================
Arturito;5430.82;35,67,80,40
Miguel;6341.47;92,77,84
Federico;10341;0,0,1,0,7
Pedrote;4500;57,68,100
\end{lstlisting}

Cada línea contiene el nombre del postulante, prioridad académica y notas. Estos tres datos están escritos en forma de string y separados por punto y coma (;), y notas está separado por comas (,).

El encargado de la selección le pide a usted que programe las siguientes funciones:
\begin{itemize}
    \item[a.] \texttt{formar\_diccionario(alumnos)} que reciba como parámetro el nombre del archivo y retorne un diccionario con los datos ordenados y separados, donde las llaves son los nombres de los postulantes y los valores son tuplas que tienen la prioridad académica y el promedio de notas.
    \begin{lstlisting}[style=consola]
>>> [* formar_diccionario('alumnos.txt') *]
{'Pedrote': (4500.0, 75.0), 
'Miguel': (6341.47, 84.3), 
'Arturito': (5430.82, 55.5), 
'Federico': (10341.0, 1.6)}
    \end{lstlisting}
    \item[b.] \texttt{agregar\_alumno(nombre,prioridad,notas,archivo)} que reciba el nombre del postulante como string, la prioridad como flotante, una lista de notas enteras, el nombre del archivo y agregue apropiadamente a este último los datos ordenados como string en la última línea.
    \begin{lstlisting}[style=consola]
>>> [*agregar_alumno('Juan',4500.12,[34,12,15,8],'alumnos.txt')*]
>>> [*arch=open('alumnos.txt')*]
>>> [*for linea in arch:*]
	[*print linea,*]

Arturito;5430.82;35,67,80,40
Miguel;6341.47;92,77,84
Federico;10341;0,0,1,0,7
Pedrote;4500;57,68,100
Juan;4500.12;34,12,15,8
>>> [*arch.close()*]
    \end{lstlisting}
    \item[c.] \texttt{modificar\_alumno(nombre,prioridad,promedio,archivo)} que modifique en \texttt{archivo} los datos del ayudante \texttt{nombre}. Use archivos temporales
    
    \begin{lstlisting}[style=consola]
>>> [*modificar_alumno('Federico',10341,[0,0,10,0,70],'alumnos.txt')*]
>>> [*arch=open('alumnos.txt')*]
>>> [*for linea in arch:*]
	[*print linea,*]
	
Arturito;5430.82;35,67,80,40
Miguel;6341.47;92,77,84
Federico;10341;0,0,10,0,70
Pedrote;4500;57,68,100
Juan;4500.12;34,12,15,8
>>> [*arch.close()*]
    \end{lstlisting}
    \item[d.] \texttt{calcular\_puntaje(prioridad,promedio)} que reciba dos enteros y calcule el puntaje correspondiente. Imagine que ordena los datos como puntos en un plano cartesiano donde los ejes son la prioridad y el promedio. El puntaje es la distancia desde el punto óptimo al punto solicitado.
    \begin{displaymath}
    \sqrt{(prioridad-7000)^{2}+(promedio-100)^{2}}
    \end{displaymath}
    \begin{lstlisting}[style=consola]
>>> [*calcular_puntaje(7900.0,70.0)*]
900.4998611882181
    \end{lstlisting}
    \item[e.] \texttt{seleccionar\_ayudante(archivo)} Esta función elige del archivo de alumnos al postulante con menor puntaje (calculado en base a la función anterior) y retorna el nombre del alumno
    \begin{lstlisting}[style=consola]
>>> [*seleccionar_ayudante('alumnos.txt')*]
'Miguel'
    \end{lstlisting}
\end{itemize}

\textbf{Ayuda}: Recuerde que los buenos programizadores usan las funciones anteriores en las funciones nuevas del programa. 

\textbf{Ayuda 2}: En los ejemplos lo que está marcado con negritas son los códigos que puede usar para probar sus funciones en la consola de Python.