\section{Código Morse}

Usted está programando un robot que pretende realizar un viaje interestelar. Su intuición le dice que debería enseñarle a transformar números normales del sistema decimal a código morse (sistema de representación de letras y números mediante señales intermitentes).

Para ello, usted creará una función \texttt{transformar(numero)} que reciba un número de tipo \texttt{int} y retorne una cadena o \texttt{str} con el código morse respectivo, donde \texttt{o} es punto y \texttt{-} es raya.

\begin{table}[h]
\centering
\begin{tabular}{|l|l|l|l|}
\hline
\textbf{1} & o - - - -  & \textbf{6}  & - o o o o \\ \hline
\textbf{2} & o o - - -  & \textbf{7}  & - - o o o \\ \hline
\textbf{3} & o o o - -  & \textbf{8}  & - - - o o \\ \hline
\textbf{4} & o o o o -  & \textbf{9}  & - - - - o \\ \hline
\textbf{5} & o o o o o  & \textbf{10} & - - - - - \\ \hline
\end{tabular}
\end{table}

\begin{lstlisting}[style=consola]
>>> [*transformar(7)*]
'--ooo'

>>> [*for numero in range(10):*]
	[*print numero, transformar(numero)*]

	
0 -----
1 o----
2 oo---
3 ooo--
4 oooo-
5 ooooo
6 -oooo
7 --ooo
8 ---oo
9 ----o
\end{lstlisting}