\section{Basquetbol}
En el básquetbol existen tres diferentes tipos de anotaciones:
\begin{itemize}
    \item el tiro libre (L), que vale un punto
    \item el doble (D), que vale dos puntos, y
    \item el triple (T), que vale tres puntos
\end{itemize}}
Un partido de básquetbol está dividido en tres períodos.

Usted debe escribir un programa que reciba como entrada una única línea, que contenga todas las anotaciones realizadas por un equipo de básquetbol durante un partido. Las anotaciones de períodos distintos deben ir separadas por un espacio. Como salida, debe mostrar la cantidad de puntos obtenidos en cada período y los puntos totales, siguiendo el formato del ejemplo.

\begin{lstlisting}[style=consola]
Anotaciones: [*DDTDLLDD DDLDT TDTLLD DDDDD*]
15 puntos en el periodo 1
10 puntos en el periodo 2
12 puntos en el periodo 3
10 puntos en el periodo 4
Total: 47 puntos
\end{lstlisting}