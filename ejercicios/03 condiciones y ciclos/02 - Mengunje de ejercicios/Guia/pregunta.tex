\begin{itemize}
    \item Un programa que pida el nombre del usuario, si es la mañana, la tarde o la noche y lo salude respectivamente:
    \begin{lstlisting}[style=consola]
    Ingrese su nombre: Juan
    Ingrese momento del dia: tarde
    Hola Juan, ¡Buenas Tardes!
    \end{lstlisting}
    
    \item Un programa que pida las notas de 3 certamenes e informe del promedio de ellos, si el usuario aprueba (promedio
    mayor a 55), si se va a global (promedio mayor a 45), o si reprueba:
    \begin{lstlisting}[style=consola]
    Ingrese primer certamen: 5
    Ingrese segundo certamen:9.7
    Ingrese tercer certamen:-4
    El promedio de los certamenes es 3.567
    El usuario reprueba.
    \end{lstlisting}
    
    \item Un programa que, recibiendo una hora específica, indique cuanto tiempo falta para otra hora ingresada en formato HH:MM
    \begin{lstlisting}[style=consola]
    Ingrese las horas actuales:17
    Ingrese los minutos actuales:45
    Ingrese las horas siguientes:23
    Ingrese los minutos siguientes:00
    Son las 17:45 y para que sean las 23:00 queda 05:15 
    \end{lstlisting}
    Note que ahora la hora se entrega con 2 digitos tanto para las horas como para los minutos.
    
    \item Un programa que pida un numero entero y indique si es par o no.
    \begin{lstlisting}[style=consola]
    Ingrese un numero: 57
    El numero 57 es impar
    
    Ingrese un numero: 4999392
    El numero 499392 es par
    \end{lstlisting}
    \item Haga un programa que reciba 2 palabras, que indique cual de ellas es mas larga, y por cuanto lo es.
    \begin{lstlisting}[style=consola]
    Ingrese la primera palabra: Gato
    Ingrese la segunda palabra: Hipopotamo
    La palabra Hipopotamo es mas larga por 6 letras
    \end{lstlisting}
    
    \item
    Cuando la Tierra completa una órbita alrededor del Sol, no han transcurrido exactamente 365 rotaciones sobre sí misma, sino un poco más. Más precisamente, la diferencia es de más o menos un cuarto de día.
Para evitar que las estaciones se desfasen con el calendario, el calendario juliano introdujo la regla de introducir un día adicional en los años divisibles por 4 (llamados bisiestos), para tomar en consideración los cuatro cuartos de día acumulados.
Sin embargo, bajo esta regla sigue habiendo un desfase, que es de aproximadamente 3/400 de día.
Para corregir este desfase, en el año 1582 el papa Gregorio XIII introdujo un nuevo calendario, en el que el último año de cada siglo dejaba de ser bisiesto, a no ser que fuera divisible por 400.
Escriba un programa que indique si un año es bisiesto o no, teniendo en cuenta cuál era el calendario vigente en ese año:
    \begin{lstlisting}[style=consola]
Ingrese un anno: 1988
1988 es bisiesto

Ingrese un anno: 2011
2011 no es bisiesto

Ingrese un anno: 1700
1700 no es bisiesto

Ingrese un anno: 1500
1500 es bisiesto

Ingrese un anno: 2400
2400 es bisiesto
    \end{lstlisting}

\end{itemize}