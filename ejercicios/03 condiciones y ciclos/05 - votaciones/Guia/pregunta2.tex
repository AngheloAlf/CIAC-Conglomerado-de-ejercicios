\section{Votaciones}

  La CONFECH,
  en su afán por agilizar el proceso de votaciones
  le ha encargado el desarrollo de un programa
  de registro de votación por universidades.
  
  Primero,
  el programa debe solicitar al usuario
  ingresar el número de universidades que participan
  en el proceso.
  Luego,
  para cada una de las universidades,
  el usuario debe ingresar el nombre de la universidad
  y los votos de sus alumnos,
  que pueden ser:
  \emph{aceptar} (A),
  \emph{rechazar} (R),
  \emph{nulo} (N) o
  \emph{blanco} (B).
  El término de la votación se realiza ingresando 
  la letra X,
  tras lo cual se debe mostrar los totales de los votos
  de la universidad,
  con el formato que se muestra en el ejemplo.
  
  Finalmente,
  el programa debe mostrar el resultado de la votación,
  indicando la cantidad de universidades
  que aceptan,
  que rechazan
  y en las que hubo empate entre ambas opciones.
  
  \begin{lstlisting}[style=consola]
  Numero de universidades: [*3*]
 
  Universidad: [*USM*]
  Voto: [*A*]
  Voto: [*R*]
  Voto: [*A*]
  Voto: [*N*]
  Voto: [*X*]
  USM: 2 aceptan, 1 rechazan, 0 blancos, 1 nulos.

  Universidad: [*UChile*]
  Voto: [*A*]
  Voto: [*B*]
  Voto: [*A*]
  Voto: [*X*]
  UChile: 2 aceptan, 0 rechazan, 1 blancos, 0 nulos.

  Universidad: [*PUC*]
  Voto: [*A*]
  Voto: [*R*]
  Voto: [*R*]
  Voto: [*A*]
  Voto: [*X*]
  PUC: 2 aceptan, 2 rechazan, 0 blancos, 0 nulos.

  Universidades que aceptan: 2
  Universidades que rechazan: 0
  Universidades con empate: 1
  \end{lstlisting}