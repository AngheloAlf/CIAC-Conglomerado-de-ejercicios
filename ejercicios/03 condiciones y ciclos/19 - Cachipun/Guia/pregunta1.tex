%%%Pregunta 1

\section{Factorial}

El factorial de un numero $n$ se denota como $n!$, lo que es igual a  $1*2*3*...*n$. Sabiendo esto haga un diagrama de flujo reciba como entrada un numero $n$ y escriba cual es el factorial de ese numero.

%%%Pregunta 2
\section{Ecuacion cuadratica}

La ecuacion cuadratica esta definida por:
\begin{align*}
    f(x) &= y = ax^2 + bx + c
\end{align*}

Su punto de vertice esta dado segun las siguientes ecuaciones:
\begin{align*}
    x_0 &= \frac{-b}{2a} \\
    y_0 &= f(x_0) = a\left(\frac{-b}{2a}\right)^{2} + b\left(\frac{-b}{2a}\right) + c
\end{align*}

Las raíces de la ecuación cuadrática se calculan con la formula:
\begin{align*}
    x &= \frac{-b \pm \sqrt{b^2 - 4ac}}{2a}
\end{align*}

Sabiendo esto escriba un programa que reciba los valores $a$, $b$ y $c$ e indique cual es el vértice y las raíces de esta ecuación, además de imprimir la ecuación cuadrática.

\begin{lstlisting}[style=consola]
    Ingrese a: [*2*]
    Ingrese b: [*8*]
    Ingrese c: [*-24*]
    
    El punto vertice de la ecuacion es (-2.0, -32.0)
    Las raices de la ecuacion son: 2.0 y -6.0
    La ecuacion cuadratica es: 2.0x^2 + 8.0x + -24.0
\end{lstlisting}

Nota: Tenga en consideración que ocurre cuando el determinante es mayor, menor o igual a cero.

