\section{El Supermercado CIAC}

Existe un supermercado llamado Centro Importante de Alimentos Comestibles (CIAC), en donde existen 3 vendedores estrellas
\begin{itemize}
    \item Miguel
    \item Anghelo
    \item Paolo
\end{itemize}

Ellos llevan la contabilidad diaria en un programa de Python, pero en un acto bandálico, robaron el computador donde tenían el código.
\\ \\   
Por eso le pide a usted, estudiante dedicado con el ramo IWI131, que programe un sistema de contabilidad. Este es un programa simple (o eso creen), existen 4 opciones

\begin{enumerate}
    \item \textbf{Ingresar venta:} Pide al usuario el nombre del vendedor y el monto vendido para guardarlos.
    \item \textbf{Mejor vendedor:} Indica quien ha sido el vendedor que más dinero ha recaudado.
    \item \textbf{Resumen diario:} Imprime por pantalla el monto total vendido por día, y te da las buenas noches. Esta opción limpia las variables para que, el siguiente día se pueda tener otro mejor vendedor y otro monto.
    \item \textbf{Salir:} Termina el programa
\end{enumerate}

El funcionamiento es el siguiente

\begin{lstlisting}[style=consola]
1. Ingresar venta
2. Mejor vendedor
3. Resumen diario
4. Salir

Ingrese opcion: [*1*]
Ingrese nombre vendedor [Miguel/Anghelo/Paolo]: [*Miguel*]
Ingrese monto vendido: [*4500*]
Ingrese opcion: [*1*]
Ingrese nombre vendedor [Miguel/Anghelo/Paolo]: [*Anghelo*]
Ingrese monto vendido: [*2000*]
Ingrese opcion: [*2*]
El mejor vendedor es Miguel, que ha vendido $4500
Ingrese opcion: [*3*]
Hoy se han vendido $6500
Buenas noches!
Ingrese opcion: [*4*]
>>>
\end{lstlisting}