\section*{Conversor}

  La Oficina de Asuntos Internacionales de la Universidad ha hecho un buen trabajo
  generando convenios de intercambio con diferentes universidades de Estados Unidos.
  Uno de los programas más recientes tiene el objetivo de realizar un intercambio,
  de alumnos de primer año, durante el segundo semestre.
  
  Para realizar la postulación,
  la universidad de destino pide un resumen de notas de los postulantes,
  pero dicho formato es diferente al usado en la USM,
  por lo que una conversión es necesaria.
  
  Las conversiones obedecen a las siguientes reglas:
  \begin{itemize}
  \item Si \(nota \ge 90\), se obtiene una nota convertida A.
  \item Si \(nota \ge 70\) y \(< 90 \), se obtiene una nota convertida B.
  \item Si \(nota \ge 55\) y \(< 70 \), se obtiene una nota convertida C.
  \item Si \(nota < 55\), se obtiene una nota convertida F.
  \end{itemize}
  
  Para confeccionar el resumen de notas,
  además de las notas convertidas,
  se necesita el promedio ponderado de las notas convertidas
  llamado GPA según sus siglas en inglés.
  
  Para calcular el GPA,
  primero se procede a convertir las notas
  según la siguiente tabla de conversión.
  
  \begin{table}[H]
    \begin{tabular}{c|c|c|c|c}
      \toprule
      Nota EE.UU. & A & B & C & F \\
      \midrule
      Puntos GPA  & 4.0 & 3.0 & 2.0 & 0.0  \\
      \bottomrule
    \end{tabular}
  \end{table}
  
  Después de obtener los puntos GPA correspondientes a cada asignatura,
  se procede a calcular el promedio ponderado por los créditos de las 2 asignaturas
  con mejor nota (PGPA),
  entre los ramos "Programación" (3 créditos),
  "Matemática I" (5 créditos) e "Introducción a la Física" (3 créditos).
  
  \begin{equation*}
    PGPA = \frac{GPA_1 \cdot C_1 + GPA_2 \cdot C_2}{C_1 + C_2}
  \end{equation*}
  
  Donde \(GPA_1\) y \(GPA_2\) corresponden a las 2 mejores notas como puntos GPA
  y \(C_1\) y \(C_2\) a sus créditos respectivamente.
    
  \pagebreak[4]
  
  Ahora usted debe:
  
  \begin{itemize}
  \item[a)]
    Desarrollar la función \texttt{convertir\_eeuu(nota)}
    que recibe como parámetro una nota en escala 0 -- 100 y retorne la nota
    en formato EE.UU.
    
    \begin{lstlisting}[style=consola]
    >>> convertir_eeuu(80)
    'B'
    \end{lstlisting}
  \item[b)]
    Desarrollar la función \texttt{convertir\_gpa(nota)}
    que recibe como parámetro una nota en formato EE.UU.
    y retorne la nota en puntos GPA.
    
    \begin{lstlisting}[style=consola]
    >>> convertir_gpa('F')
    0.0
    \end{lstlisting}
  \item[c)]
    Escriba un programa en el cual se ingresen las notas
    de las 3 asignaturas mencionadas e imprima en pantalla
    las notas convertidas al formato de EE.UU.
    y el PGPA.
    
    \begin{lstlisting}[style=consola]
    Nota de Programacion: [*80*]
    EE.UU: B
    Nota de Matematica: [*60*]
    EE.UU: C
    Nota de Fisica: [*95*]
    EE.UU: A
    PGPA: 3.5
    \end{lstlisting}
  \end{itemize}