\section{Polinomios}
  Un polinomio de grado \(n\)
  es una función matemática de la forma
  \begin{equation*}
    p(x) = a_0
      + a_1 \cdot x
      + a_2 \cdot x^2
      + \hdots
      + a_n \cdot x^n
  \end{equation*}
  Los valores \(a_0, \, a_1, \, \hdots \, , \, a_n\)
  son los coeficientes del polinomio,
  y \(x\) es la variable independiente.
  Desarrolle un programa que evalúe un polinomio.
  
  Primero,
  el usuario debe ingresar \(x\).
  A continuación,
  debe ingresar los coeficientes en orden.
  Para indicar que todos los coeficientes han sido ingresados,
  se debe escribir el texto \texttt{FIN}.
  Finalmente,
  el programa debe mostrar el valor de \(p(x)\).
  El ejemplo muestra como evaluar el polinomio:
  \(p(x) = -7 - 3x^2 + 2x^3\) en \(x = 2.1\).
  
  \begin{lstlisting}[style=consola]
  x: [*2.1*]
  Coeficientes:
  [*-7*]
  [*0*]
  [*-3*]
  [*2*]
  [*FIN*]
  p(x) = -1.708
  \end{lstlisting}
  
  \pagebreak[4]