\section{Armandose el tarro}

  Pedro es un comprador informado.
  Cada vez que necesita renovar su computador
  averigua las características principales del equipo
  y decide si comprarlo o no siguiendo un criterio 
  de puntaje.
  Los criterios utilizados por pedro son los siguientes:
  
  \begin{itemize}
  \item
    \textbf{RAM}.
    Pedro considera que un computador con 4 [GB] de memoria
    corresponde a 2 puntos.
    A los computadores con menos memoria les asigna 1 puntos
    y a los que tienen más,
    3 puntos.
  \item
    \textbf{HDD}.
    Para la capacidad del disco duro,
    Pedro considera discos entre 
    500 [GB] y 1 [TB] con 2 puntos.
    A los discos más pequeños les asigna 1 puntos
    y a los más grandes,
    3 puntos.
  \item
    \textbf{Precio}.
    Pedro piensa que un computador es caro
    si cuesta más de un millón de pesos,
    a los que les asigna 1 punto.
    Los computadores mayores o iguales a quinientos mil
    son considerados con 2.
    Si el precio es menor,
    Pedro asigna 3 puntos
  \end{itemize}

  Pedro quiere comprar un computador para montar
  un \emph{Media Center},
  es decir como apoyo para ver videos y películas en un
  televisor.
  Por esta razón el factor más importante en la decisión es
  el disco duro con un peso del 50 \%,
  seguido por el precio (40\%) y,
  finalmente,
  la capacidad de la memoria (10\%).
  Pedro solo comprará computadores basados en estas reglas
  que obtengan un puntaje mayor a 2.
  
  Escriba un programa en Python que solicite al usuario
  la cantidad de memoria,
  disco duro y precio.
  Luego debe mostrar por pantalla el puntaje del computador
  y si califica para comprarse.
  
  
  \begin{lstlisting}[style=consola]
  Bienvenido Usuario!
  ingrese cantidad de memoria RAM (GB): [*3*]
  ingrese cantidad de disco duro (GB) : [*600*]
  ingrese precio (pesos)              : [*800000*]
  El puntaje del computador es 1.9
  No se sugiere comprarlo!
  \end{lstlisting}
  
  \begin{lstlisting}[style=consola]
  Bienvenido Usuario!
  ingrese cantidad de memoria RAM (GB): [*8*]
  ingrese cantidad de disco duro (GB) : [*1200*]
  ingrese precio (pesos)              : [*1200000*]
  El puntaje del computador es 2.2
  Vale la pena comprarlo!
  \end{lstlisting}
  
  \pagebreak[4]