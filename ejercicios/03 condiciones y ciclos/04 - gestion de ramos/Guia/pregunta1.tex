\section{Gestión de ramos}

  En la institución educacional ``Aprendiendo''
  se necesita modernizar sus sistemas de gestión de cursos.
  Para ello requieren un programa que permita registrar nuevos alumnos
  e ingresar sus notas de evaluación,
  para luego indicar su promedio final y si aprobó el ramo.
  Se le pide realizar una versión prototipo de este programa
  para el ramo de ``Programación de Computadores I'',
  el cual cuenta con 4 certámenes que siguen la siguiente
  ponderación:
  
  \begin{equation*}
  \text{Nota} = 
    C_1 \cdot 0.2 + 
    C_2 \cdot 0.2 +
    C_3 \cdot 0.3 +
    C_4 \cdot 0.3
  \end{equation*}
  
  El programa debe mostrar dos opciones,
  una para ingresar alumno y sus notas
  y la otra para salir del programa,
  tal como se muestra a continuación:
  
  \begin{lstlisting}[style = consola]
  ***************************************
  * Bienvenido al sistema de evaluacion *
  * Programacion de Computadores I      *
  ***************************************
  Opciones:
  1) Ingresar Alumno
  2) Salir
  Que desea hacer?: [*1*]
  
  Nombre alumno: [*Pablo Inzunza*]
  Nota C1: [*55*]
  Nota C2: [*60*]
  Nota C3: [*67*]
  Nota C4: [*40*]
  El alumno Pablo Inzunza aprobo con 55 en promedio.
  
  Opciones:
  1) Ingresar Alumno
  2) Salir
  Que desea hacer?: [*1*]
  
  Nombre alumno: [* Andres Fuentes *]
  Nota C1: [*30*]
  Nota C2: [*40*]
  Nota C3: [*40*]
  Nota C4: [*55*]
  El alumno Andres Fuentes reprobo con 43 en promedio.
  
  Opciones:
  1) Ingresar Alumno
  2) Salir
  Que desea hacer?: [*2*]
  
  Adios!
  \end{lstlisting}
\pagebreak[4]