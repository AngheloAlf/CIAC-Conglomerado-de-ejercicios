\documentclass[spanish, fleqn]{scrartcl}
\usepackage[utf8]{inputenc}
\usepackage{babel}
\usepackage[headheight=47pt, paper=a4paper, top=3cm, left=2cm, right=2cm]{geometry}
\usepackage{tikz}
\usepackage{CIACcustom}
\usepackage{fourier}
\usepackage{amsmath, amsthm}
\usepackage{listings}
\usepackage{multicol}
\usepackage{fancyhdr}
\usepackage[urlcolor=blue, colorlinks]{hyperref}
\usepackage{booktabs,tabularx}
\usepackage{float}

\newcolumntype{L}[1]{>{\hsize=#1\hsize\raggedright\arraybackslash}X}%
\newcolumntype{R}[1]{>{\hsize=#1\hsize\raggedleft\arraybackslash}X}%
\newcolumntype{C}[2]{>{\hsize=#1\hsize\columncolor{#2}\centering\arraybackslash}X}%

\pagestyle{fancy}
\fancyhf{}
\rhead{\pgfimage[width=2.5cm]{imagenes/logo-ciac.png}}
\chead{
  Apoyos Intensivos Pauta Tutor N° 1\\
  IWI-131 Semestre I-2017 \\
  CIAC Casa Central
}
\lhead{\pgfimage[width=2.5cm]{imagenes/logo-usm.jpg}}
\rfoot{\LaTeXe / CIAC 2017}
\lfoot{\thepage}

\renewcommand{\ttdefault}{pcr}

%%% listings settings:
\definecolor{bggray}{rgb}{0.95,0.95,0.95}
\lstdefinestyle{consola}{
  backgroundcolor=\color{bggray},
  basicstyle=\small\ttfamily,
  frame=single,
  moredelim=[is][\bfseries]{[*}{*]},
  xrightmargin=5pt
}

\lstdefinestyle{mypy}{
  language=python,
  backgroundcolor=\color{bggray},
  basicstyle=\ttfamily\small\color{orange!70!black},
  frame=L,
  keywordstyle=\bfseries\color{green!40!black},
  commentstyle=\itshape\color{purple!40!black},
  identifierstyle=\color{blue},
  stringstyle=\color{red},
  numbers=left,
  showstringspaces=false,
  xrightmargin=5pt,
  xleftmargin=10pt
}

\newtheorem{CIACdef}{Definición}

\begin{document}
\vspace*{-0.4cm}
\section*{Pauta Tutor - Intensivo Programación}

\begin{itemize}
    \item 19:00 -   10-15 minutos para presentación
    \begin{enumerate}
        \item Nombres, carreras y tiempo trabajando
        
        \item Objetivo de los intensivos e historia, en el caso de progra el intensivo es de tipo taller, excepcionalmente se harán diapositivas
        
        \item Reglas del juego (Chiste por preguntar si el programa corre teniendo el computador al frente)
        
        \item No hay problema con llegar atrasado, pero tendrá que esperar por la guía
        
        \item La asistencia es voluntaria, no anotar a sus amigos en las listas, no salir sin estar en la lista
        
        \item Hablar de apoyos personalizados y posible BNA
        
        \item Python es el lenguaje más fácil para aprender algoritmos
    \end{enumerate}
        
    \item 19:15 -   20-30 minutos para diapositivas de diagrama de flujo, funciones y condiciones
    
    \item 19:35 -   40-50 minutos para trabajo individual/grupal con consultas
    
    \item 20:20 -   10 minutos para retirar computadores y resolver preguntas finales
\end{itemize}

\section{Manejo de strings}

  \paragraph{Solución}
  A continuación se expone el código de los ejercicios de la sección.
  
  \lstinputlisting[
    style  = mypy,
    caption= \texttt{Pauta}]{Pauta/1.py}
    
\section{Manejo de datos numéricos}

    \paragraph{Solución}
    A continuación se expone el código de los ejercicios de la sección, por motivos de espacio la frase final del ejercicio 2 se dividió en 3 variables.
    
    \lstinputlisting[
    style =mypy,
    caption= \texttt{Pauta}]{Pauta/2.py}

\section{Entrada a la Disco}

    \paragraph{Solución}
    A continuación se expone el código del ejercicio, tenga en cuenta que pueden existir más formas, pero aqui se ejemplifica el uso de todas las estructuras condicionales disponibles en un bloque.
    
    \lstinputlisting[
    style =mypy,
    caption= \texttt{Pauta}]{Pauta/3.py}
    
\section{Código Morse}

    \paragraph{Solución}
    A continuación se expone el código del ejercicio, tenga en consideración que se pide programar una función, por lo que debe existir un retorno del tipo string, y no una impresión por pantalla.
    
    \lstinputlisting[
    style =mypy,
    caption= \texttt{Pauta}]{Pauta/4.py}
    

\end{document}
