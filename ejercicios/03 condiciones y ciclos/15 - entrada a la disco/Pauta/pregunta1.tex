
\section{Manejo de strings}

  \paragraph{Solución}
  A continuación se expone el código de los ejercicios de la sección.
  
  \lstinputlisting[
    style  = mypy,
    caption= \texttt{Pauta}]{Pauta/1.py}
    
\section{Manejo de datos numéricos}

    \paragraph{Solución}
    A continuación se expone el código de los ejercicios de la sección, por motivos de espacio la frase final del ejercicio 2 se dividió en 3 variables.
    
    \lstinputlisting[
    style =mypy,
    caption= \texttt{Pauta}]{Pauta/2.py}

\pagebreak[4]

\section{Entrada a la Disco}

    \paragraph{Solución}
    A continuación se expone el código del ejercicio, tenga en cuenta que pueden existir más formas, pero aqui se ejemplifica el uso de todas las estructuras condicionales disponibles en un bloque.
    
    \lstinputlisting[
    style =mypy,
    caption= \texttt{Pauta}]{Pauta/3.py}
    
\section{Código Morse}

    \paragraph{Solución}
    A continuación se expone el código del ejercicio, tenga en consideración que se pide programar una función, por lo que debe existir un retorno del tipo string, y no una impresión por pantalla.
    
    \lstinputlisting[
    style =mypy,
    caption= \texttt{Pauta}]{Pauta/4.py}
    