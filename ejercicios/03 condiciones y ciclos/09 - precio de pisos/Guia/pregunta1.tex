\section{Precio de pisos}
  Un edificio tiene 25 pisos
  de 8 departamentos  cada  uno.
  La dueña del edificio ha definido una estrategia
  para ponerle precio a cada departamento.
  El número que identifica cada departamento 
  se divide en dos partes: 
  los dos últimos dígitos indican en qué posición
  está (de acuerdo al diagrama),
  y los restantes  indican  el  piso.
  Por ejemplo, el departamento 1105
  está en el undécimo piso,
  en la posición 5.
  
  \begin{center}
  \pgfimage{imagen/1.png}
  \end{center}
  
  Los dos departamentos al extremo derecho del diagrama
  tienen vista al mar,
  y los dos del extremo izquierdo tienen vista al cerro.
  Todos los departamentos del primer piso valen 100,
  y todos los departamentos del último piso valen 400.
  Para los pisos intermedios, 
  se ha fijado un precio base de 245;
  el precio de los departamentos con vista al mar
  se aumentará en 13,
  y el de los con vista al cerro 
  se rebajará en 17.
  Los decimales se redondearán hacia abajo.
  Adicionalmente,
  se difundió el rumor de que el
  ídolo adolescente Justino Vivar
  habría alojado una noche en el departamento 807.
  Como hay un gran interés entre sus fanáticas
  por adquirir este departamento,
  la dueña ha decidido fijar su precio en 500.

  Escriba un programa que pregunte al comprador
  el número del departamento,
  y le entregue cuál es el precio de ese departamento.
