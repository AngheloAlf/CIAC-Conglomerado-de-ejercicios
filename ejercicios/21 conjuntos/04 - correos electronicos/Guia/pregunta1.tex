\section{Correos electrónicos}

El dominio de un correo electrónico es la parte que va después del @. Suponiendo que \texttt{correos} es un diccionario cuyas llave son correos electrónico y sus valores son tuplas de la forma \texttt{(CorreosRecibidos, CorreosEnviados)} programe:

\begin{lstlisting}[style=consola]
correos={'miguel.godoy.13@sansano.usm.cl':(1315,165),
         'magodoydiaz@gmail.com':(1126,268),
         'cristiano.ronaldo@gmail.com':(34202,45),
         'donfrancisco@yaju.es':(2450003,301),
         'federico.santama@alumnos.usm.cl':(315,8)}
\end{lstlisting}

\begin{itemize}
    \item[a.] Una función \texttt{dominios(correos)} que reciba el diccionario \texttt{correos} y retorne un conjunto con los dominios existentes.
    
    \item[b.] Una función \texttt{total(correos,seImprime)} que reciba el diccionario \texttt{correos} y un valor booleano si se desea imprimir, y retorne un diccionario donde la llave es el correo y el valor es la suma de los correos recibidos y enviados. Esta función imprimirá el total global de correos, es decir la suma de todos los valores si \texttt{seImprime} es True.
    \begin{lstlisting}[style=consola]
>>> total(correos,True)
Correos totales:  2487748
{'miguel.godoy.13@sansano.usm.cl': 1480,
'federico.santama@alumnos.usm.cl': 323,
'magodoydiaz@gmail.com': 1394,
'donfrancisco@yaju.es': 2450304,
'cristiano.ronaldo@gmail.com': 34247}
    \end{lstlisting}
    
    \item[c.] Una función \texttt{mayorUso(correos)} que reciba el diccionario \texttt{correos} y retorne el correo electrónico que más correos haya enviado y recibido.
    
    \begin{lstlisting}[style=consola]
>>> mayorUso(correos)
'donfrancisco@yaju.es'
    \end{lstlisting}
\end{itemize}