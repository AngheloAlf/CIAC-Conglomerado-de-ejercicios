\section{Sansano Airport}

  El aeropuerto internacional SansanoAirport desea
  mantener la información acerca de las salidas
  y llegadas de sus distintos vuelos;
  para ello,
  a implementado un sistema de información
  basado en Python,
  utilizado,
  desarrollado y mantenido por Sansanos.
  La base de dicho sistema
  es un \texttt{diccionario}
  llamado \texttt{SansaVuelos}
  el cual almacena la información
  de cada uno de los vuelos.
  
  La clave del \texttt{diccionario}
  es un \texttt{String} que representa el código
  del vuelo y el valor del \texttt{diccionario}
  es una \texttt{tupla} compuesta respectivamente
  por el \textbf{nombre} de la aerolínea,
  el lugar de \textbf{partida},
  \textbf{destino},
  una lista de \texttt{String} con las \textbf{escalas}
  y el \textbf{tiempo promedio}
  de todos los vuelos realizados entre partida y destino
  (medido en minutos).
  
  \begin{lstlisting}[style=mypy]
    SansaVuelos = {
      # codigo: (nombre, partida, destino, escalas, tiempo_promedio),
      'NH217': ('All Nippon Airways','Tokyo','Santiago', ['Atlanta'],1620),
      'AY154': ('Finnair','Helsinki','Moscu', ['Riga'], 175),
      'OV171': ('Estonian Air','Tallin','Paris', ['Amsterdam','Berlin'], 205)}
  \end{lstlisting}
  
  También se cuenta con un \texttt{diccionario Vuelo},
  cuya clave es un entero \textbf{identificador},
  y cuyo valor es una \texttt{tupla}
  compuesta respectivamente por el \textbf{código}
  del vuelo y una \textbf{fecha}
  (en forma de tupla (año, mes, día)).
  
  \begin{lstlisting}[style=mypy]
    Vuelo = {
      # identificador: codigo, fecha
      3542: ('AY154', (2013,12,25)),
      6433: ('NH217', (2013,12,31)),
      1313: ('NH217', (2013,11,6))}
  \end{lstlisting}
  
  \begin{itemize}
  \item[a)]
    Escriba la función 
    \texttt{vuelos\_entre\_fechas(fecha1,fecha2,Vuelo)}
    que reciba un par de fechas y \texttt{Vuelo},
    y retorne el \texttt{conjunto} de todos los
    códigos de vuelos que se hayan realizado entre esas
    dos fechas.

    \begin{lstlisting}[style=consola]
      >>> vuelos_entre_fechas((2013,7,22),(2014,7,22),Vuelo)
      set(['AY154', 'NH217'])
    \end{lstlisting}
  \item[b)]
    Programe 
    \texttt{vuelos\_agotadores(fecha1,fecha2,Vuelos, SansaVuelos)}
    que reciba un par de fechas y los \texttt{diccionarios}
    \texttt{Vuelo} y \texttt{SansaVuelos}
    y retorne un \texttt{conjunto}
    con los códigos de los vuelos realizados entre dichas fechas
    (ambas inclusive) y que sean considerados como \emph{agotadores}.
    Un vuelo es considerado como agotador si tiene tres o más escalas
    o si su tiempo promedio de vuelo es igual o superior a 8 horas.

    \begin{lstlisting}[style=consola]
      >>> vuelos_agotadores((2013,7,22),(2014,7,22),Vuelo,SansaVuelos)
      set(['NH217'])
    \end{lstlisting}
  \item[c]
    Escriba la función
    \texttt{vuelos(partida,destino, Vuelo, SansaVuelos)}
    que recibe un par de \texttt{Strings}
    con el nombre de la ciudad de partida
    y la ciudad de destino y los
    \texttt{diccionarios}
    \texttt{Vuelo} y \texttt{SansaVuelos}.
    Esta función debe retornar una lista de tupĺas,
    en donde cada tupla está compuesta por el \textbf{código}
    del vuelo,
    el \textbf{nombre} de la aerolínea
    y una lista con las fechas en las que se ha
    efectuado dicho viaje.
    
    \begin{lstlisting}[style=consola]
      >>> vuelos('Helsinki','Moscu', Vuelo, SansaVuelos)
      [('AY154', 'Finnair', [(2013, 12, 25)])]
    \end{lstlisting}
  \end{itemize}