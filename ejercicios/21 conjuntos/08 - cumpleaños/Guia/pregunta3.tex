\section{Cumpleaños}
Se tiene un diccionario \texttt{fechas} cuya llave es el nombre de un ser viviente y su valor es una tupla de la forma \texttt{anio,mes,dia} que indica la fecha de nacimiento del personaje.

\begin{lstlisting}[style=consola]
fechas={
    'Mike':(1994,7,19),
    'Gary':(1989,5,22),
    'Brad':(1975,5,22),
    'Angie':(1984,5,22),
    'Peter':(1967,12,4),
    'Larry':(2001,3,14),
    'Moe':(2000,12,4)
    #...
    }
\end{lstlisting}

Se le pide a usted programar las siguientes funciones:

\begin{itemize}
\item[a.] \texttt{masViejo(fechas)} que reciba el diccionario \texttt{fechas} y retorne el nombre del personaje con más edad.

\begin{lstlisting}[style=consola]
>>> [*masViejo(fechas)*]
'Peter'
\end{lstlisting}

\item[b.] \texttt{primerCumple(fechas)} que reciba el diccionario \texttt{fechas} y retorne el nombre del personaje cuyo cumpleaños es el primero del año.

\begin{lstlisting}[style=consola]
>>> [*primerCumple(fechas)*]
'Larry'
\end{lstlisting}

\item[c.] \texttt{quienesCompartenCumple(fechas)} que reciba el diccionario \texttt{fechas} y retorne un diccionario cuya llave sea una tupla de la forma \texttt{(mes,dia)} y su valor un conjunto con los nombres de los personajes que tienen cumpleaños el mismo día. Note que no importa el año en que nacieron, y que el diccionario que retorna no puede tener algún elemento cuyo conjunto (asociado al valor) tenga sólo un personaje.

\begin{lstlisting}[style=consola]
>>> [*quienesCompartenCumple(fechas)*]
{(5, 22): set(['Brad', 'Angie', 'Gary']), (12, 4): set(['Peter', 'Moe'])}
\end{lstlisting}

\end{itemize}