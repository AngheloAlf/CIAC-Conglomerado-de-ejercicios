\section{Rey programador}

Para este problema es importante hacer un esquema de lo que se quiere lograr con cada función auxiliar. 
\begin{itemize}
    \item Se crea un conjunto con el nombre de todos los posibles candidatos, para preguntar individualmente cuantas generaciones de programadores tiene.
    \item Seguido de esto, se crea un diccionario nuevo, con la misma estructura, pero ahora la llave es el nombre de un hijo y el valor un conjunto de padres.
    \item Se crea una función ascendencia para calcular cuantas generaciones tiene cierta persona.
    \item Finalmente, se crea un algoritmo de optimización, donde se retorna una lista con los mejores candidatos y se quitan los elementos que están en el conjunto de \texttt{muertos}.
\end{itemize}

\lstinputlisting[
    style = mypy,
    caption = \texttt{pauta\_rey.py}]{Code/pauta_rey.py}
