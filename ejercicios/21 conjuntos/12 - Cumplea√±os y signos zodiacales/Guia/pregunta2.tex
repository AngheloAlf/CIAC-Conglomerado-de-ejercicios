\section{Cumpleaños + Signos zodiacales}
Luego de las noticias de que la NASA, en su intento por figurar en el muro de los más débiles, había cambiado las fechas de los signos zodiacales, Pedro Engel sintió la necesidad de programar algunas utilidades para un grupo de amigos.

Suponga que se tiene un diccionario \texttt{signos} con la información de las fechas límites de cada signo, donde la llave es el signo y el valor es una tupla de tuplas de la forma \texttt{(fechaInicio,fechaFin)}.
\begin{lstlisting}[style=consola]
signos = {
   'aries':       (( 3, 21), ( 4, 20)),
   'tauro':       (( 4, 21), ( 5, 21)),
   'geminis':     (( 5, 22), ( 6, 21)),
   'cancer':      (( 6, 22), ( 7, 23)),
   'leo':         (( 7, 24), ( 8, 23)),
   'virgo':       (( 8, 24), ( 9, 23)),
   'libra':       (( 9, 24), (10, 23)),
   'escorpio':    ((10, 24), (11, 22)),
   'sagitario':   ((11, 23), (12, 21)),
   'capricornio': ((12, 22), ( 1, 20)),
   'acuario':     (( 1, 21), ( 2, 19)),
   'piscis':      (( 2, 20), ( 3, 20)),
}
\end{lstlisting}
Se tiene también un diccionario \texttt{fechas} cuya llave es el nombre de un ser viviente y su valor es una tupla de la forma \texttt{anio,mes,dia} que indica la fecha de nacimiento del personaje.

\begin{lstlisting}[style=consola]
fechas={
    'Mike':(1994,7,19),
    'Gary':(1989,5,22),
    'Brad':(1975,5,22),
    'Angie':(1984,5,22),
    'Peter':(1967,12,4),
    'Larry':(2001,3,14),
    'Moe':(2000,12,4)
    #...
    }
\end{lstlisting}

Pedrito Engel le pide a usted programar las siguientes funciones:

\begin{itemize}
\item[a.] \texttt{queSigno(fecha,signos)} que reciba una fecha con el formato de tupla \texttt{(anio,mes,dia)} y el diccionario \texttt{signos} y retorne un string con el signo correspondiente.

\begin{lstlisting}[style=consola]
>>> [*queSigno((1994,7,19),signos)*]
'cancer'
>>> [*queSigno((1994,12,23),signos)*]
'capricornio'
\end{lstlisting}

\item[b.] \texttt{NombresMasSigno(fechas,signos)} que reciba los diccionarios \texttt{fechas} y \texttt{signos} y retorne una lista de tuplas, estas últimas deben ser de la forma \texttt{nombre,signo} asociando el nombre del personaje del diccionario \texttt{fechas} con su signo respectivo

\begin{lstlisting}[style=consola]
>>> [*NombresMasSigno(fechas,signos)*]
[('Mike', 'cancer'), ('Angie', 'geminis'), ('Gary', 'geminis'), 
('Larry', 'piscis'), ('Brad', 'geminis'), ('Peter', 'sagitario')]
\end{lstlisting}

\item[c.] \texttt{quienesCompartenCumple(fechas)} que reciba el diccionario \texttt{fechas} y retorne un diccionario cuya llave sea una tupla de la forma \texttt{(mes,dia)} y su valor un conjunto con los nombres de los personajes que tienen cumpleaños el mismo día. Note que no importa el año en que nacieron, y que el diccionario que retorna no puede tener algún elemento cuyo conjunto (asociado al valor) tenga sólo un personaje.

\begin{lstlisting}[style=consola]
>>> [*quienesCompartenCumple(fechas)*]
{(5, 22): set(['Brad', 'Angie', 'Gary']), (12, 4): set(['Peter', 'Moe'])}
\end{lstlisting}

\end{itemize}