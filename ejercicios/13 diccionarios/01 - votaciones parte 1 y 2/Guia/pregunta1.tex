\section{Votaciones Parte 1}
Para ciertas votaciones no secretas de cierto grupo, se tiene un diccionario que asocia el nombre del votante con su decisión (1:afirmativa, 0:negativa)

\begin{lstlisting}[style=consola]


\end{lstlisting}

\begin{itemize}
    \item[a.] Cree una función \texttt{separar\_nombres(votaciones)} que retorne una tupla con dos listas, la primera con los nombres de los que votaron si, y la segunda con los que votaron no.
    \item[b.] Cree una función \texttt{porcentaje\_si(votaciones)} que retorne el porcentaje de votos afirmativos que hubo en el acto.
    \item[c.] Cree una función \texttt{se\_acepta(votaciones)} que retorne un booleano con respecto a la decisión tomada en las votaciones.
\end{itemize}

\section{Votaciones Parte 2}

Dentro de las decisiones que toma a cabo este grupo, se lleva a votación la cantidad de tazas de café que se pueden tomar mientras debaten, donde el mínimo es 0 (ninguna) y el máximo es 10. Para esta votación no se tomaron los nombres, sólo sus votos, los cuales se ordenaron en forma de lista:

\begin{lstlisting}[style=consola]
tazas =[1,5,6,3,4,5,1,3,4,6,8,5,3,7,9,0,7,5,4,6,0,9,6,4,6,7,3,2]
\end{lstlisting}

Se le pide a usted

\begin{itemize}
    \item[a.] Desarrollar la función \texttt{votos\_por\_cantidad(tazas)} que retorne un diccionario que asocie la cantidad de tazas con la cantidad de votos que obtuvo.
    \begin{lstlisting}[style=consola]
>>> votos_por_cantidad(tazas)
{0: 2, 1: 2, 2: 1, 3: 4, 4: 4, 5: 4, 6: 5, 7: 3, 8: 1, 9: 2}
    \end{lstlisting}
    \item[b.] Desarrollar la función \texttt{moda(tazas)} que retorne la cantidad de tazas más votada
    \item[c.] Programar la función \texttt{promedio(tazas)} que retorne una cantidad entera correspondiente al promedio de tazas resultante de la votación.
\end{itemize}