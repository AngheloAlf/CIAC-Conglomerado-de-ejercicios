\section{Pukemones}

  El profesor Pine es un reconocido científico
  del mundo Pukeman y tiene en su haber
  una cantidad considerable de Pukemanes.
  Como buen científico,
  el profesor Pine es muy ordenado
  y tiene a los Pukemanes almacenados
  en un diccionario,
  con sus respectivas características,
  de la siguiente manera:
  
  \begin{lstlisting}[style = consola]
  Pukemanes = {
    # nombre: (puntos de salud, ataque, defensa,
               at. especial, def.especial, velocidad, tipo),
    'Bulbasaur':(45, 49, 49, 65, 65, 45,'Grass'),
    'Charmander':(39, 52, 43, 60, 50, 65,'Fire'),
    'Pikachu':(35, 55, 40, 50, 50, 90,'Electric'),
    'Jiglypuff': (115, 45, 20, 45, 25, 20,'Normal'),
    # ...
  }
  \end{lstlisting}
  
  El profesor Pine recibe la visita de 3 amigos:
  Nash, fisty y Block.
  El profesor les da a elegir
  un Pukeman a cada uno.
  Dado que cada uno tiene una estrategia distinta
  para elegir,
  y que la cantidad de Pukemanes es muy grande,
  Ud. tiene que ayudarles escribiendo
  una función que retorne al
  (los) Pukeman(es) de acuerdo a sus demandas.
  
  \begin{itemize}
  \item[a)]
    Nash prefiere a los Pukemanes equilibrados,
    por lo que debe escribir una función
    que retorne al Pukeman cuya varianza de las características
    numéricas sea la menor.
    La varianza la puede calcular de la siguiente manera:
    \(var(x) = \left( \frac{1}{N} \sum x^2 \right) - \overline{x}^2\),
    donde \(\overline{x}\) es el promedio.
    
    \begin{lstlisting}[style = consola]
    >>> mejor_Nash(Pukemanes):
    'Charmander'
    \end{lstlisting}
  \item[b)]
    Fisty tiene una predilección por los
    Pukemanes de Césped
    (tipo \texttt{Grass})
    y por los que la suma de su ataque especial
    y defensa especial sea la mayor.
    
    \begin{lstlisting}[style = consola]
    >>> mejor_Fisty(Pukemanes)
    'Bulbasaur'
    \end{lstlisting}
  \item[c)]
    Block es más difícil de satisfacer.
    El no cree mucho en las estadísticas
    por lo que prefiere elegir a su Pakeman
    de acuerdo a sus instintos.
    Pero claro,
    como todos tiene sus preferencias.
    El prefiere a los Pukemanes de tipo
    \texttt{Normal} y \texttt{Electric},
    y además que sus puntos de salud sea
    mayor o igual a cierto valor
    (valor dado como parámetro).
    Ayúdele a Block filtrando a los Pukemanes,
    generando otro diccionario donde los Pukemanes
    agregados cumplan con esas características.
    
    \begin{lstlisting}[style=consola]
    >>> filtro_Block(Pukemanes, 35)
    {'Pikachu': (35, 55, 40, 50, 50, 90, 'Electric'),
    'Jiglypuff': (115, 45, 20, 45, 25, 20, 'Normal')}
    \end{lstlisting}
  \end{itemize}