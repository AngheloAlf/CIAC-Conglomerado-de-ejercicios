%%%Pregunta 1
\section{Deudas}

Como buen ser humano y generoso que es usted, siempre que sus amigos tienen problemas de dinero, usted les presta. Pero como ellos son tan irresponsables con el dinero, ellos nunca tienen la misma cantidad de dinero para pagar y le terminan \textit{abonando} las deudas.

Como sus amigos le siguen pidiendo dinero prestado y apenas pagan, se esta volviendo loco intentando llevar la cuenta de cuanto dinero le deben, así que ha decidido crear un programa en python que lo ayude.

Para esto, ha creado 2 diccionarios. El diccionario deudas lleva por llave el rut de la persona que le debe dinero, y por valor tiene una lista, la cual contiene muchas listas que describen en que fecha le presto dinero a esta persona y cuanto le presto.

\begin{lstlisting}[style=consola]
deudas = {
# rut: [ [(AAAA, MM, DD), dinero], ... ]
"19125145-8": [[(2016, 8, 26), 5200], [(2016, 9, 26), 4200],
 [(2016, 10, 26), 9800]],
"19222444-6": [[(2017, 3, 5), 3250], [(2017, 5, 3), 8200]], 
"18336423-9": [[(2015, 5, 29), 8400], [(2016, 8, 5), 14000],
 [(2017, 3, 26), 16000]]}
\end{lstlisting}

El diccionario abonos indica cuando la persona le abono dinero que le debía y cuanto dinero le abono.

\begin{lstlisting}[style=consola]
abonos = {
"19125145-8": [[(2016, 10, 03), 1200], [(2016, 10, 18), 2500],
 [(2016, 12, 14), 7000]],
"19222444-6": [[(2017, 5, 12), 600], [(2017, 5, 19), 1000],
 [(2017, 5, 21), 2500]], 
"18336423-9": [[(2015, 9, 23), 2000], [(2017, 2, 15), 3000]]}
\end{lstlisting}

Para poder hacer un programa que lo pueda ayudar con su problema cree las siguientes funciones:

\begin{enumerate}

\item deudaPorRut(deudas, rut); la cual recibe el diccionario de deudas y un rut. Retorna la cantidad total que debe la persona en cuestión, sin contar sus abonos.

\begin{lstlisting}[style=consola]
>>> deudaPorRut(deudas, "19125145-8")
19200
\end{lstlisting}

\item deudaTotalRut(deudas, abonos, rut); la cual recibe el diccionario de deudas, el de abonos y un rut. Retorna cuanto debe la persona en cuestión, pero se le resta lo que ya ha abonado.

\begin{lstlisting}[style=consola]
>>> deudaTotalRut(deudas, abonos, "19125145-8")
8500
\end{lstlisting}

\item dinero(deudas, abonos); la cual recibe el diccionario de deudas y el de abonos. Retorna cuanto le deben en total todos sus amigos.

\begin{lstlisting}[style=consola]
>>> dinero(deudas, abonos)
49250
\end{lstlisting}

\item agregarDeuda(deudas, rut, fecha, monto); la cual recibe el diccionario de deudas, el rut del deudor, una tupla que contiene una fecha, y un entero que indica el monto de la deuda. Debe retornar el diccionario deudas que contiene la nueva deuda. 

\item agregarAbono(abonos, rut, fecha, monto); la cual recibe el diccionario de abonos, un rut, una tupla que contiene una fecha, y un entero que indica el monto del abono. Debe retornar el diccionario de abonos que contiene el nuevo abono. 


\end{enumerate}


