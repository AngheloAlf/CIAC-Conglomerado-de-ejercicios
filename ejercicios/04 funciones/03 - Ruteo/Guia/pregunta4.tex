%%% Pregunta 2
\section{Números binarios}

Un numero binario es una secuencias de 1s y 0s que pueden representar distintas cosas dependiendo de su uso. Uno de sus usos es representar números decimales (los números que usamos a diario, del 0 al 9) como secuencia de 1s y 0s.

Para saber a que número representa una secuencia binaria, hay que multiplicar cada numero de los 1s y 0s de la secuencia, de forma invertida, por potencias de dos. Explicado de otra forma, invertir la secuencia, multiplicar el primer numero (que puede ser tanto un 1 o un 0), por 2 elevado a 0, luego el siguiente numero por 2 elevado a 1, y así sucesivamente hasta completar la secuencia.

Sabiendo esto, escriba la función \texttt{binario\_a\_numero(binario)}, que reciba una secuencia de 1s y 0s, en formato string y retorne el número al cual representa. 

\begin{lstlisting}[style=consola]
>>> [*binario_a_numero("11001")*]
25
\end{lstlisting}

\begin{lstlisting}[style=consola]
>>> [*binario_a_numero("001011")*]
11
\end{lstlisting}
