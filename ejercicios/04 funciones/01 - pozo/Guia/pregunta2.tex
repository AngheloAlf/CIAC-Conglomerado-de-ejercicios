\section{Definición de funciones}

  La compañía Posos Ltda.
  Se dedica a la excavación y utilización de napas subterráneas de agua.
  Como últimamente han tenido un aumento considerable
  en la demanda de personas que quieren excavar posos en sus propiedades,
  la empresa ha decidido automatizar el cálculo del costo
  de los posos con el fin de hacer del trámite de cotización
  más expedito.
  Para facilitar el trabajo de los administradores de sistemas
  de la empresa,
  se le pide que programe el software por partes
  utilizando funciones sencillas.
  A continuación programe las funciones que se le soliciten:
  
  \begin{enumerate}
  \item
    Programe la función \texttt{areaPozo(diametro, profundidad)}
    que reciba el diámetro y profundidad esperados del pozo a construir,
    y retorne el área correspondiente a las paredes del pozo
    (el cilindro sin ambas tapas).
    Por simplicidad considere \( \pi = 3.14\).
    
    \begin{lstlisting}[style=consola]
    [*>>> areaPozo(2,3)*]
    18.84
    \end{lstlisting}
  \item
    Programe la función \texttt{volumenPozo(diametro, profundidad, nivel)}
    que reciba los mismos parámetros anteriores más el nivel de agua y retorne
    el volumen de agua en el pozo.
    El parámetro \texttt{nivel} indica a cuántos metros está el agua de la superficie,
    por lo que el volumen de agua se calcula utilizando como altura del cilindro
    \texttt{profundidad - nivel}.
    
    \begin{lstlisting}[style=consola]
    [*>>> volumenPozo(2,3,1)*]
    6.28
    \end{lstlisting}
  \item
    Escriba un programa que calcule el costo de excavar el pozo utilizando las funciones
    anteriormente programadas.
    El programa debe pedir por pantalla los datos del pozo
    es decir diámetro requerido, profundidad deseada y nivel de agua esperado.
    Luego debe calcular el costo considerando como costos los siguientes datos:
    
    \vspace*{\baselineskip}
    
    \begin{tabular}{c|c}
      \toprule
      Costo excavación & \$ 1000 por metro de profundidad \\
      \midrule
      Pared del pozo   & \$ 100 por metro cuadrado de pared \\
      \midrule
      inscripción      & \$ 50 por metro cúbico de agua en el pozo \\
    \bottomrule
    \end{tabular}

    \vspace*{\baselineskip}
    
    \begin{lstlisting}[style=consola]    
    Posos Ltda
    Bienvenido al calculador de costo
    Ingrese el diametro del poso a construir: [*2*]
    Ingrese la profundidad de la excavacion : [*3*]
    Ingrese el nivel de profundidad de agua : [*1*]
    El costo de excavacion corresponde a: $ 5198.0
    \end{lstlisting}
  \end{enumerate}