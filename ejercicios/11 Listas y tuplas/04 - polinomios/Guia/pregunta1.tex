\section{Polinomios}

  Un polinomio de grado \(n\)
  es una función matemática de la forma:
  
  \begin{equation*}
    p(x) = a_0 + a_1 \cdot x + a_2 \cdot x^2 + \hdots + a_n \cdot x^n
  \end{equation*}
  
  Donde \(x\) es el parámetro y
  \(a_0, a_1, \hdots, a_n\) son números reales dados.
  Algunos ejemplos de polinomios son:
  \begin{itemize}
  \item \(p(x) = 1 + 2x + x^2\)
  \item \(q(x) = 4 - 17x\)
  \item \(r(x) = -1 -5x^3 + 3x^5\)
  \item \(s(x) = 5x^{40} + 2x^{80}\)
  \end{itemize}
  
  Los grados de estos polinomios son, respectivamente,
  \(2\), \(1\), \(5\) y \(80\).
  
  Evaluar un polinomio significa reemplazar \(x\)
  por un valor y obtener el resultado de la función.
  Por ejemplo, si evaluamos el polinomio \(p\)
  en el valor \(x = 3\),
  obtenemos el resultado:
  
  \begin{equation*}
    p(3) = 1 + 2 \cdot 3 + 3^2 = 16
  \end{equation*}
  
  Un polinomio puede ser representado como una lista
  con los valores \texttt{a\_0}, \texttt{a\_1}, ..., \texttt{a\_n}.
  Por ejemplo, los polinomios anteriores
  pueden ser representados en un programa como sigue:
  
  \begin{lstlisting}[style=consola]
  >>> p = [1, 2, 1]
  >>> q = [4, -17]
  >>> r = [-1, 0, 0, -5, 0, 3]
  >>> s = [0] * 40 + [5] + [0] * 39 + [2]
  \end{lstlisting}
  
  \begin{enumerate}
  \item 
    Escriba la función \texttt{grado(p)}
    que entregue el grado de un polinomio dado:
    \begin{lstlisting}[style=consola]
    >>> grado(r)
    5
    >>> grado(s)
    80
    \end{lstlisting}
  \item
    Escriba la función \texttt{evaluar(p, x)}
    que evalúe el polinomio \(p\) 
    (representado como una lista)
    en el valor \(x\):
    \begin{lstlisting}[style=consola]
    >>> evaluar(p, 3)
    16
    >>> evaluar(q, 0.0)
    4.0
    >>> evaluar(r, 1.1)
    -2.82347
    >>> evaluar([4, 3, 1], 3.14)
    23.2796
    \end{lstlisting}
  \item
    Escriba la función \texttt{sumar\_polinomios(p1, p2)}
    que entregue la suma de dos polinomios:
    \begin{lstlisting}[style=consola]
    >>> sumar_polinomios(p, r)
    [0, 2, 1, -5, 0, 3]
    \end{lstlisting}
  \item
    Escriba la función \texttt{derivar\_polinomio(p)}
    que entregue la derivada del polinomio:
    \begin{lstlisting}[style=consola]
    >>> derivar_polinomio(r)
    [0, 0, -15, 0, 15]
    \end{lstlisting}
  \item
    Escriba la función \texttt{multiplicar\_polinomios(p1, p2)}
    que entregue el producto entre los polinomios de entrada:
    \begin{lstlisting}[style=consola]
    >>> multiplicar_polinomios(p, q)
    [4, -9, -30, -17]
    \end{lstlisting}
  \end{enumerate}
  \pagebreak[4]