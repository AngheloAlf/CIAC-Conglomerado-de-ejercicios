\section{La montaña más alta del mundo.}

Francis Galegs y Michael God, famosos escaladores, programadores y jóvenes empresarios internacionales, sedientos de nuevas aventuras, deciden hacer trabajar a los estudiantes de programación que van a los intensivos del CIAC para saber cual es la montaña más alta que han escalado.

La información de sus aventuras la guardan en una lista de tuplas con el nombre \texttt{montanas} que tiene el formato \texttt{nombre,altura,hora\_inicio,hora\_fin}, donde \texttt{nombre} es un string, \texttt{altura} es un entero en metros y los datos restantes son tuplas de la forma \texttt{(horas,minutos)}.
\\
\begin{lstlisting}[style=consola]
montanas=[
    ('Nielol',335,(14,00),(17,35)),
    ('La Campana',1880,(8,4),(17,34)),
    ('Churen Himal',7385,(5,34),(23,14)),
    ('Everest',8848,(8,23),(10,43)), #fue en helicoptero
    ('Mont Blanc',4810,(4,12),(21,45)),
    #...
    ('Kilimanjaro',5895,(12,15),(19,31))
    ]

\end{lstlisting}
\\
Ellos quieren que los estudiantes programen las siguientes funciones

\begin{itemize}
    \item \texttt{filtro\_altura\_nombre(montanas)} que reciba la lista \texttt{montanas} y retorne una lista de tuplas con la altura y el nombre ordenadas de menor a mayor.
    \\
    \begin{lstlisting}[style=consola]
>>> filtro_altura_nombre(montanas)
[(335, 'Nielol'), (1880, 'La Campana'), (4810, 'Mont Blanc'), 
(5895, 'Kilimanjaro'), (7385, 'Churen Himal'), (8848, 'Everest')]
    \end{lstlisting}

    \item \texttt{duracion(inicio,fin)} que reciba dos tuplas con la hora de inicio y fin y retorne una tupla de la forma \texttt{hora,minutos} con la duración de la escalada
    \\
    \begin{lstlisting}[style=consola]
>>> duracion((14,0),(17,35))
(3, 35)
    \end{lstlisting}
    \item Finalmente escriba un programa que indique cual fue la montaña más alta que escalaron, la altura en metros que tenía y el tiempo empleado. La información que se muestra por pantalla debe tener la siguiente forma
    \begin{lstlisting}[style=consola]
La montana mas alta que escalaron fue Everest.
Esta tenia una altura de 8848 metros.
Se demoraron 2 horas y 20 minutos.
    \end{lstlisting}
    
\end{itemize}