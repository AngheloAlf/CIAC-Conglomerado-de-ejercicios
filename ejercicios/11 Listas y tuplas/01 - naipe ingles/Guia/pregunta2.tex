% \pagebreak[4]
\section{Listas y Tuplas}

  Eduardo es aficionado a los juegos de cartas
  utilizando naipe inglés.
  Como no tiene mucho tiempo para reunirse con sus amigos
  a jugar debido a su horario de trabajo decide
  que sería una buena idea jugar en línea.
  Para ello solicita una herramienta de apoyo
  para gestionar la baraja utilizando Python.
  
  \paragraph{Ejercicio}
  Para ayudar a Eduardo resuelva los siguientes problemas.
  
  \begin{enumerate}
  \item
    Programe la función \texttt{nueva\_baraja()}
    que retorne una lista con tuplas de la forma
    \texttt{(palo, numero)},
    por ejemplo \texttt{('trebol', '2')}.
  \item
    Programe la función \texttt{robar(baraja)}
    que retorne los datos de la primera carta de la baraja.
  \item
    Programe la función \texttt{barajar(baraja)}
    que retorne una nueva baraja a partir de la primer,
    con las cartas alineadas al azar
    (pista: puede utilizar
    funciones aleatorias como \texttt{random} y \texttt{choice}.
  \item
    Programe la función \texttt{entregar\_mano(baraja, cartas)}
    que retorne una lista con las primeras cartas de la baraja.
    El número de cartas robadas corresponde a la variable \texttt{cartas}.
  \item
    Escriba un programa que cree una baraja y pida un número de jugadores.
    Para cada jugador,
    el programa debe barajar las cartas y entregar una mano,
    luego debe entregar el total de cartas de cada palo en la mano.
    
    \begin{lstlisting}[style=consola]
    Ingrese numero jugadores: [*1*]
    Ingrese numero cartas por mano: [*5*]
    mano jugador 1:
    pica: 3
    corazon: 1
    diamante: 1
    trebol: 0
    \end{lstlisting}
  \end{enumerate}